\section{Conclusion}
\label{sec:conclusion}
This proposal has outlined a clear market demand, a novel research gap, and a robust technical methodology for developing a next-generation AI companion. The literature review demonstrated that current research is bifurcated, with SOTA frameworks focusing on either purely functional task-completion \cite{AgentOrchestra, ui-venus} or the fundamentals of relational AI \cite{EmotionAWARE, Four-Quadrant-Taxonomy}. A significant gap exists for a unified system that blends both.

To address this gap, this project will design, build, and validate a novel, hierarchical agent architecture. This system, managed by a central ``Conductor'' and integrated via an MCP bus, is designed for high performance and stability. It leverages a hybrid-latency functional core for real-time action, a proactive relational core for companionship, and a robust safety core for error recovery. This architecture will be validated through a rigorous, two-phased development plan on complex, high-DOF games.

The primary contribution of this work is not just the prototype itself, but the creation and validation of an architectural blueprint for this new class of ``companion-style assistant''. By empirically testing the hypothesis that functional competence is a prerequisite for a strong relational bond \cite{Cooperation-Player-AI}, this project will provide a significant and timely contribution, demonstrating a prototype capable of the deeper, functional partnership that defines the future of interactive AI.