\section{Conclusion}
\label{sec:conclusion}
This proposal has outlined a clear market demand, a novel research gap, and a robust technical methodology for developing a next-generation AI companion. As the literature review demonstrated, current research is \textbf{largely polarized} between SOTA frameworks for functional task-completion and those for relational AI \cite{AgentOrchestra, ui-venus, EmotionAWARE, Four-Quadrant-Taxonomy}. This \textbf{creates a clear gap} for a unified system that can effectively blend both.

To address this gap, this project will design, build, and validate a novel, hierarchical agent architecture. This system, managed by a central ``Conductor'' and integrated via an MCP bus, is designed for high performance and stability. It leverages a hybrid-latency \textbf{Functional Core} for real-time action, a proactive \textbf{Relational Core} for companionship, and a robust \textbf{Safety Core} for error recovery. This architecture will be validated through a rigorous, two-phased development plan on complex, high-DOF games.

The primary contribution of this work is not just the prototype itself, but the creation and validation of an architectural blueprint for this new class of ``companion-style assistant''. By empirically testing the hypothesis that functional competence is a prerequisite for a strong relational bond, this project will provide a significant and timely contribution, demonstrating a prototype capable of the deeper, functional partnership that defines the future of interactive AI \cite{Cooperation-Player-AI}.