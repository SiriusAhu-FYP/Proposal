\section{Literature Review}
% > 说明:按taxonomy组织;每小节末有“与本文关系”。Cradle放在2.2(GUI/GCC)。涉及了2.1-2.9。

\subsection{Perception: Modalities \& Grounding}
% > 输入模态与定位。涉及2.3-2.4。
\todo{视觉为主(screen/video)+ 可选音频(audio);VLM能力:检测/描述/grounding;可简单对照VLA(直接产出action tokens)与VLM+tool的差别。
与本文关系:本节仅界定术语与能力范畴。}

\subsection{Action Interfaces: GUI (GCC) \& MCP-style Orchestration}
% > 说明:以GUI(GCC)作为统一执行通道;MCP作为内部“技能总线(skill bus)”的注册/编排思想(不讨论API适配)。Cradle在此作GUI可行性的代表。涉及2.6-2.7。

在动作接口上,\textbf{GUI} 路线以 \textbf{General Computer Control (GCC)} 为统一通道(\emph{screen-in, keyboard/mouse-out}),强调跨应用/跨游戏的可迁移性(portability)与人类同态交互(human-homomorphic interface)。代表性工作 \emph{Cradle} 展示了在不依赖应用专用接口的前提下,通过\emph{规划—技能整理(skill curation/registry)—反思—记忆}的管线完成长链路任务(desktop/games),为 \emph{GUI} 可行性提供了实证支持。与此同时,\textbf{MCP}(Model Context Protocol)提供了\emph{模块注册/编排(module registration/orchestration)}的协议化思路:在不改变输出仍为 \emph{GUI} 的条件下,\emph{skills/macros、planning、memory、reflection} 等可在统一接口下组织,便于可复现实验与消融比较。\cite{tan2024cradle,park2025orak}

\paragraph{Takeaway}
文献显示:\emph{GUI(GCC)}提供统一接口与较低移植门槛;协议化编排(如\emph{MCP})有助于模块化与复现性。与本文关系:本节作为动作接口背景与术语界定。

…Cradle 以统一的 GUI(GCC)通道展示了从“屏幕输入→内在推理→键鼠控制”的闭环(见 \autoref{fig:cradle-overview})。

\picHere{./assets/images/from-papers/cradle01.jpg}{0.9\linewidth}
{An overview of the CRADLE framework: CRADLE takes video from the computer screen as input and outputs computer keyboard and mouse control determined through inner reasoning (planning, skill curation, reflection, memory) \cite{tan2024cradle}.}
{fig:cradle-overview}

\paragraph{UI-Venus(screenshot-only)}
端到端 GUI 导航,无需 planner/A11y;强调\emph{截图$\to$结构化动作}的通道在真实平台可达到具有竞争力的结果(如 AndroidWorld \,pass@1 与 ScreenSpot 系列定位)\cite{ui-venus}。与本文关系:作为文献实例表明 screenshot-only 路线的可行性。

\subsection{Agentic Modules: Planning, Memory, Reflection, Skills}
% > 机制视角。涉及2.1, 2.2, 2.4。
\todo{规划(planning)、记忆(memory, 用户偏好/历史)、反思(self-reflection, 纠错/风格一致)、技能库(skills/macros, 原子→复合)。
与本文关系:仅作机制分类与代表性做法的回顾。}

\paragraph{历史对齐与稀疏动作增强}
提出 \emph{Self-Evolving Trajectory History Alignment \& Sparse Action Enhancement}:
用当前模型重写历史“思维—动作”轨迹以对齐风格/细节,并上采样稀疏但关键动作(如 LongPress),以改善长链路一致性与泛化\cite{uivenus_rft}。与本文关系:作为处理长链路与长尾动作的文献做法。

\subsection{Learning Paradigms: Zero-shot, RAG, Finetune, IL/RL, Distillation}
% > 训练与推理范式。涉及2.1, 2.3。
\todo{零样本/提示工程、检索增强(RAG for UI schema/FAQ)、轻量微调(LoRA)、模仿/强化(IL/RL)、蒸馏到小模型。与本文关系:范式综述,不含实现承诺。}

以“\emph{指令化}(instructionalization)”增强 RL 代理的上下文理解是一条代表性路线。\emph{R2-Play} 将多模态游戏指令(MGI)并入 \emph{Decision Transformer}(\emph{DTGI}),并通过超网络(\emph{SHyperGenerator})在训练任务与未见任务间共享知识;作者报告多模态指令较文本/轨迹单模态在多任务与泛化上更优(动机见 \autoref{fig:r2play-motivation}),MGI 的三段式结构——\emph{game description}、\emph{game trajectory}、\emph{game guidance}(含动作、语言引导及关键元素位置)——给出了指令模板(见 \autoref{fig:r2play-mgi})\cite{r2play}。与本文关系:作为“指令化/Decision Transformer”方向的文献背景。

\picHere{./assets/images/from-papers/r2-play01.jpg}{0.7\linewidth}
{Imagine an agent learning to play Palworld. (1) The agent exhibits confusion when only relying on textual guidance. (2) The agent is confused when presented with images of a Pal sphere and a Pal. (3) The agent understands how to catch a pet through \emph{multimodal guidance}, which combines textual guidance with images of the Pal sphere and Pal \cite{r2play}.}
{fig:r2play-motivation}

\picHere{./assets/images/from-papers/r2-play02.jpg}{0.9\linewidth}
{An illustrative example of \emph{multimodal game instructions (MGI)}. Each instruction consists of three sections: \emph{game description}, \emph{game trajectory}, and \emph{game guidance} (including action, language guidance, and the position of key elements) \cite{r2play}.}
{fig:r2play-mgi}

\paragraph{RFT(GRPO)与动作粒度奖励}
将奖励拆分为\emph{格式/动作类型/坐标/内容}四部分并加权,以同时度量\emph{结构化输出合规性}与\emph{细粒度定位/文本输入}正确性,作为 GUI 导航中 RL-finetune 的代表做法之一\cite{uivenus_rft}。与本文关系:作为 RL-finetune 在 GUI 导航中的奖励设计范例。

\subsection{Benchmarks \& Datasets (OS-like, Games, Desktop)}
% > 三段法——Orak(统一评测/消融/MCP思想)→ lmgame-Bench(脚手架+污染控制)→ Procedural-generation(OOD方法学)。涉及2.6。每段末“与本文关系”一句,不作采用承诺。

\paragraph{Orak(统一评测/消融/MCP思想)}
\emph{Orak} 通过 \emph{MCP} 实现\emph{plug-and-play} 的代理—环境解耦,并在统一配置下检验
\emph{planning / reflection / memory / skills} 等\emph{agentic modules} 的边际贡献(ablation),配套
\emph{Leaderboard/Battle Arena} 与训练轨迹数据(fine-tuning trajectories),将\emph{机制—性能—配置}一体化呈现\cite{park2025orak}。
与本文关系:作为“统一评测与消融”的代表性基准。

\paragraph{Procedural-generation(OOD方法学)}
基于\emph{procedural generation} 的开放式评测在可控生成下构造\emph{OOD} 与多步任务压力,比较
\emph{VLA/VLM} 在\emph{架构/训练数据/输出后处理}等变量下的泛化与稳健性,并配套工具链以保证\emph{reproducibility}\cite{guruprasad2025benchmarking}。
与本文关系:作为 OOD/变量可控的评测方法学背景。

\paragraph{lmgame-Bench(脚手架与污染控制)}
\emph{lmgame-Bench} 将“\emph{游戏→评测}”系统化:用 \emph{Gym-style} 接口与\emph{perception/memory scaffolds}
稳定\emph{prompt} 并剔除\emph{污染},在多模型下获得良好分离度,并通过\emph{相关性分析}展示“各游戏探测的能力混合不相同”;另报告单一游戏的 \emph{RL} 训练对\emph{未见游戏}/\emph{外部规划任务}存在迁移\cite{hu2505lmgame}。
与本文关系:作为“脚手架/污染控制/迁移观察”的评测文献。

…为减少提示方差并抑制污染,lmgame-Bench 以模块化脚手架稳定“感知—记忆—推理”的交互回路(见 \autoref{fig:lmgame-harness})。

\picHere{./assets/images/from-papers/lmgame01.jpg}{0.9\linewidth}
{lmgame-Bench uses modular harnesses—such as perception, memory, and reasoning modules—to systematically extend a model’s game-playing capabilities, enabling iterative interaction loops with a simulated game environment \cite{hu2505lmgame}.}
{fig:lmgame-harness}

\paragraph{V\!-MAGE(vision-centric, visual-only, continuous-space)}
该框架以\textbf{仅视觉输入}与连续空间的游戏环境,评测多模态模型的视觉中心能力,覆盖定位、轨迹追踪、时机、视觉记忆及更高层时序推理;其评测管线支持分离“模型/策略”,并采用 Elo 风格排名进行相对强度比较;作者报告现有模型与人类表现存在差距、常见感知错误与锚定偏差,且有限历史上下文会限制长时规划\cite{v-mage}。
与本文关系:作为视觉中心评测的代表性基准。

\picHere{./assets/images/from-papers/v-mage01.jpg}{0.9\linewidth}
{The overview of the V-MAGE benchmark, designed to evaluate vision-centric capabilities and higher-level reasoning of MLLMs across 5 free-form games with 30+ levels \cite{v-mage}.}
{fig:v-mage-overview}

\subsection{Evaluation Protocols \& Metrics}
% > 强关联本文贡献;此处仅回顾文献做法,不作实现承诺。
\todo{客观:success rate, time-to-completion, no-misclick/rollback rate, latency(voice RTT, frame→hint时间);主观:advice adoption, user satisfaction。与本文关系:术语与度量背景。}

文献中常将\emph{输出结构化与解码约束(structured output \& constrained decoding)}纳入统一协议:动作以 JSON Schema 与白名单规范,经内部路由后由 GUI 执行;并以\emph{Invalid Action Rate} 作为守门指标。对稀疏/时序敏感技能,亦有工作报告机会归一化成功率(\emph{OAS})、反应时延(\emph{RT})与每次机会尝试数(\emph{APO})等指标\cite{guruprasad2025benchmarking}。
此外,关于\emph{post-processing}(技能解码、重试/回滚)的显式记录亦见于将游戏转化为可靠评测的文献\cite{hu2505lmgame}。
文献中亦有采用 \emph{Elo-style ranking} 进行跨任务相对强度比较的做法\cite{v-mage}。

\subsection{Deployment \& Real-time Considerations}
% > 工程现实。涉及2.2, 2.6。
\todo{本地/云混合、量化(INT4/FP8)、流式解码、语音中断(barge-in)、资源占用与帧率影响。与本文关系:工程背景。}

\subsection{Safety, Permissions \& Robustness}
% > 安全边界。涉及2.2。
\todo{权限模型(whitelist, scope)、操作确认、影子模式(shadow mode)先预测后执行、回滚/急停。与本文关系:安全与鲁棒性背景。}

如 \autoref{fig:uivenus_qual} 所示,think–action 不一致(mismatch)揭示了 MLLM 的“幻觉”(hallucination)风险\cite{ui-venus}。

\picHere{./assets/images/from-papers/ui-venus01.jpg}{0.95\linewidth}
{One trace of UI-Venus on the task named MarkorDeleteAllNotes in AndroidWorld. We can observe that UI-Venus successfully achieves the goal and has the reflection ability in Step 3. However, there also exists the conflict between think and action in Step 5, remaining as a future work about how to solve MLLM’s hallucination.\cite{ui-venus}}
{fig:uivenus_qual}

\subsection{Synthesis: Trends, Gaps \& Our Niche}
% > 综述收束到本文位置;仅做趋势与缺口陈述,末尾一句“与本文关系”定位。
当前趋势是在\emph{GUI(GCC)}通道上引入\emph{协议化/模块化}编排以支撑可复现实验与消融;真实多类型游戏的统一评测(如\emph{Orak})与\emph{visual-only/continuous-space} 的视觉中心评测(如\emph{V\!-MAGE})并行发展;把\emph{脚手架/污染控制}引入评测协议(如\emph{lmgame-Bench})成为常见做法。与本文关系:本文研究定位于\emph{伴随式(companion-style)}场景的文献回顾与术语/评测背景梳理。
