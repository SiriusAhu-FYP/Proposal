\section{Literature Review}
> 说明:采用taxonomy组织而非按时间;每小节末给1-2句“与本文关系”。Cradle放到2.2(GUI-only/GCC)而非开头。涉及了2.1-2.9。

\subsection{Perception: Modalities \& Grounding}
> 说明:输入模态与定位。涉及2.3-2.4。
\todo{视觉为主(screen/video)+可选音频(audio);VLM能力:检测/描述/grounding;提一嘴VLA(如果动作权重内生)与传统VLM+tool的差别。与本文:我们选轻量VLM,优先本地(on-device)与流式ASR/TTS。}

\subsection{Action Interfaces: GUI-only (GCC) vs API/MCP}
> 说明:动作接口对比的核心小节,放Cradle。涉及2.6-2.7。
\todo{定义GCC:screen-in, keyboard/mouse-out。代表作:\textbf{Cradle}(GUI-only,skill curation/registry, reflection/memory)。优点:通用性(generality)、可迁移性(portability);缺点:确定性/延迟。API/MCP:确定性高、速度快、但依赖适配。本文策略:GUI-first + API/MCP作为加速通道,GUI兜底。附一句对比表指引。}


\paragraph{GUI-only via GCC: \emph{Cradle}}
% > 说明:用4-6句讲清机制、作者声称与局限;末尾连到本文立场。
\textbf{接口范式}:提出统一的\emph{General Computer Control (GCC)}设定(\emph{screen-in, keyboard/mouse-out}),避免应用专用API,强调通用性(generality)与可迁移性(portability)。%
\ \textbf{系统机制}:规划(planning)\,$\rightarrow$\,技能整理与注册(skill curation/registry)\,$\rightarrow$\,自我反思(self-reflection)\,$\rightarrow$\,记忆(memory)的管线,降低长链路任务的错误累积。%
\ \textbf{作者声称(claimed novelty)}:首次系统化提出\emph{GCC}作为统一接口,并展示了商业游戏与桌面软件上的长时序任务可行性(长时间连续操作)。%
\ \textbf{优势/局限}:\emph{GUI-only}具通用性强,但在确定性(determinism)、延迟(latency)与复杂UI鲁棒性上受限。%
\ \textbf{与本文关系}:我们采纳其\emph{skills/macros + reflection + memory}思想,以\emph{GUI-first}保证覆盖面,同时在可用时接入\emph{API/MCP}通道以提速与稳态(保持GUI兜底)。

\paragraph{API/MCP tool-use(简述以对比)}
API/MCP路径通过结构化接口获得高确定性与较低延迟,但需要适配成本(porting cost)与接口可得性(availability)。本文选择\emph{GUI-first + API/MCP as accelerator}的折中策略。

在动作接口上,\emph{GUI-only} 的 \emph{General Computer Control (GCC)} 强调以“screen-in, keyboard/mouse-out”的人类同态通道获取最大通用性(代表作如 \emph{Cradle}),而 \emph{API/MCP} 路径则以结构化接口换取更强的确定性与更低的延迟(例如 MCP 的\emph{plug-and-play}对接与模块化评测)。两者形成“通用性—确定性”的互补光谱:本文采取\emph{GUI-first}以保证覆盖面,并在可用时以\emph{API/MCP}加速关键路径,同时保持 GUI 兜底以应对接口缺失与版本漂移。 


\paragraph{Takeaway}
GUI-only(GCC)与API/MCP形成\emph{通用性–确定性}(generality–determinism)的互补;面向实时伴随式助手(companion-style),折中方案更契合我们的延迟与稳定性目标。

\subsection{Agentic Modules: Planning, Memory, Reflection, Skills}
> 说明:机制视角。涉及2.1, 2.2, 2.4。
\todo{规划(planning)、记忆(memory, 用户偏好/历史)、反思(self-reflection, 纠错/风格一致)、技能库(skills/macros, 原子→复合)。说明这些机制如何提升长链路成功率与体验一致性。与本文:直接采纳skills+reflection+memory组合。}

\subsection{Learning Paradigms: Zero-shot, RAG, Finetune, IL/RL, Distillation}
> 说明:训练与推理范式。涉及2.1, 2.3。
\todo{列常见范式及成本/收益:零样本与提示工程、检索增强(RAG for UI schema/FAQ)、轻量微调(LoRA)、模仿/强化(IL/RL)、蒸馏到小模型。与本文:优先零样本+RAG,必要时小规模LoRA以稳UI。}

\subsection{Benchmarks \& Datasets (OS-like, Games, Desktop)}
> 说明:基准版图。涉及2.6。
\todo{按类型分:桌面/操作系统类(如OSWorld系)、游戏/模拟器类(如ALE等)与自建任务脚本。指出覆盖能力与缺口:缺少“伴随式建议/语音互动”的评测。与本文:定义我们的小型、可复现实验设置与演示脚本。}


面向真实游戏场景的评测正在从“单一玩法/小游戏”转向“跨类型、可扩展”的统一框架:\emph{Orak} 以\emph{Model Context Protocol (MCP)}提供\emph{plug-and-play}式对接,使代理(agent)与环境的连接解耦,并在统一配置下考察\emph{planning / reflection / memory / tool-use}等\emph{agentic modules}对性能的边际贡献(ablation)。这类基准不仅有助于横向比较(不同模型/模态),也便于纵向分析(同一模型的模块策略差异),从而把“机制—性能—可复现配置(reproducibility)”串到一起。本文在伴随式场景(companion-style)中借鉴其“统一评测维度 + 模块消融”的体例,但更强调低延迟与语音互动的用户体验指标(例如\emph{advice adoption}与\emph{voice RTT})。\cite{park2025orak}


\paragraph{Placement note(放置说明)}
将 Orak 置于“Benchmarks \& Datasets”主位;在“Action Interfaces: GUI vs.\ API/MCP”小节中\emph{一句话}指出其采用 \emph{MCP} 的\emph{plug-and-play} 思路以支撑模块化评测。

\subsection{Evaluation Protocols \& Metrics}
> 说明:强烈关联本文贡献。涉及2.2, 2.7。
\todo{客观:success rate, time-to-completion, no-misclick/rollback rate, latency(voice RTT, frame→hint时间);主观:advice adoption, user satisfaction。与本文:将新增advice adoption与macro success作核心指标。}

\subsection{Deployment \& Real-time Considerations}
> 说明:工程现实。涉及2.2, 2.6。
\todo{本地/云混合、量化(INT4/FP8)、流式解码、语音中断(barge-in)、资源占用与帧率影响。与本文:给出延迟预算(如$\leq$ 500ms提示、$\leq$ 1.5s语音回路)。}

\subsection{Safety, Permissions \& Robustness}
> 说明:安全边界。涉及2.2。
\todo{权限模型(whitelist, scope)、操作确认、影子模式(shadow mode)先预测后执行、回滚/急停。与本文:作为系统必要模块。}

\subsection{Synthesis: Trends, Gaps \& Our Niche}
> 说明:综述收束到本文位置。关联全篇。
\todo{趋势:GUI-only通用→API/MCP混合确定性;机制:从对话到任务化/技能化;缺口:缺少“伴随式建议+语音”的统一评测与低延迟实现。本文niche:针对实时游戏的companion-style助手,提供可复现小型协议与演示。}
