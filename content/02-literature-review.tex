\section{Literature Review}
% 总述:遵循“讲故事”的主线——先证『能做』(接口可行性),再答『怎么比』(评测协议),继而说『怎么更稳』(agentic 模块)与『怎么学』(学习范式),最后直面『做不到哪儿』(能力边界),并在文末固定『本文术语与范围口径』(glossary)。同一文献可在多处出现,但每次只取与该节主题相关的要点;不作实现承诺。

% ==========
% 2.1 接口可行性
% ==========
\subsection{接口可行性(Interface Feasibility: GUI/GCC)}
% 作用:回答“只走 GUI(GCC),不接专用 API,能否跑通真实任务?”;为后文的评测协议与模块选择提供事实前提。
% 承上启下:本节结尾过渡到 2.2——既然能做,就要统一“怎么比”。

要把“伴随式实时助手”落到真实游戏,首先要回答:\textbf{不接专用 API、仅用 GUI(GCC, screen-in \& keyboard/mouse-out)是否可行?} 代表性工作证明,在 \textbf{不依赖应用专用接口} 的条件下,通过“信息采集—规划—技能整理(macro/skill)—自反思—记忆”的管线,\textbf{可以}在桌面/游戏场景跑通长链路任务,从而确立了 \textbf{人类同态接口(human-homomorphic interface)} 的可迁移性与可复现性\cite{CRADLE}。

进一步地,端到端实践展示了“\textbf{截图} $\rightarrow$ \textbf{结构化动作(structured output)}”的可落地路径:以受约束的动作模式替代自由文本,使执行更稳且便于审计(auditability);同时,评测框架通过感知/记忆脚手架(scaffolds)稳定交互回路,把“游戏→评测”流程工程化\cite{ui-venus,lmgame-bench}。对比研究还提出\textbf{合法动作映射}与\textbf{约束解码(constrained decoding)}以显著降低无效动作(invalid actions),为后文的评测协议埋下方法学钩子\cite{Benchmarking-VLA-VLM}。\textit{既然“能做”已被验证,接下来就需要统一“怎么比”。}

% ==========
% 2.2 评测协议
% ==========
\subsection{评测协议与实用要素(Protocols \& Practicalities)}
% 作用:回答“怎么比、控什么变量、报什么指标、如何降无效动作”;把统一协议/脚手架/动作显式化/奖励设计与指标口径形成一套可复现抓手。
% 承上启下:本节结尾过渡到 2.3——在统一协议下,谁在起作用要靠可复现的模块消融来回答。

在统一协议方面,已有研究通过 \textbf{MCP-style} 编排\textbf{解耦代理与多游戏环境},统一配置与日志,\textbf{在同一协议下做可复现的消融}以比较 \textbf{planning / reflection / memory / skills} 等模块,并配套 \textbf{leaderboard} 与 \textbf{battle arena} 维持跨任务可比性\cite{ORAK}。为减少提示方差与避免训练—测试污染,评测侧引入 Gym-style 接口与感知/记忆脚手架(harness),记录 \textbf{post-processing} 并固定环境版本,报告在多模型下具有良好分离度的结果\cite{lmgame-bench}。

在“无效动作治理”上,\textbf{动作显式化}是共同结论:将输出\textbf{映射到合法离散动作空间}并采用\textbf{结构化输出/约束解码}联合抑制 \textbf{Invalid\%} 与坐标偏差;评估层面同时使用 \textbf{Micro/Macro Precision/Recall/F1} 与 \textbf{Brier/MAE} 兼顾分类与校准\cite{Benchmarking-VLA-VLM}。与此一致,近期综述提出两条工程路径:其一将自由文本\textbf{语义映射到最近的允许动作};其二直接\textbf{在允许动作集合上建模概率},两者均可与本文“合法动作映射/约束解码”的实现口径对齐\cite{llm-ga}。面向结构化动作的端到端导航,还可在 RL-finetune 中将奖励拆分为\textbf{格式/动作类型/坐标/内容}四粒度,将“合规性与细粒度正确性”纳入可学习信号,形成\textbf{工程可落地}的闭环\cite{ui-venus}。

为了与“伴随式实时体验”对齐,本文在客观指标(\textbf{success/pass@k}、\textbf{time-to-completion, TTC}、\textbf{misclick/rollback}、\textbf{latency}——含 \textbf{voice RTT})与\textbf{macro/micro} 并报之外,引入三项\textbf{机会导向}工作定义:\textbf{OAS}(\textbf{opportunity-normalized success},成功次数÷可操作机会数)、\textbf{RT}(\textbf{reaction time per opportunity},机会出现到首个有效动作/提示的时间)与\textbf{APO}(\textbf{attempts per opportunity},每次机会的平均尝试/回滚次数)\cite{ORAK,lmgame-bench,Benchmarking-VLA-VLM,ui-venus}。

% ==========
% 2.3 Agentic 模块
% ==========
\subsection{Agentic 模块(Planning / Memory / Reflection / Skills)}
% 作用:把长链路稳定性的结构要素(planning/memory/reflection/skills)与“可复现消融”和“训练侧配套”放在同一叙事里。
% 承上启下:本节结尾过渡到 2.4——让这些模块更稳/更强,学习范式是关键杠杆。

围绕长链路稳定性,研究通常将代理拆解为可组合的\textbf{四要素}:\textbf{planning}(任务分解与策略选择)、\textbf{memory}(短长时与用户偏好)、\textbf{self-reflection}(纠错与风格一致)与 \textbf{skills/macros}(原子到复合技能)。该组合有助于缓解错误累积与状态漂移,支撑 \textbf{GCC} 通道上的可迁移闭环\cite{CRADLE};在统一协议下,\textbf{可复现的模块消融}提供了边际效应与搭配选择的证据,使“用/不用、强/弱”不再停留于经验判断\cite{ORAK}。训练与数据侧通常与\textbf{结构化动作}相匹配:通过 \textbf{RL-finetune} 与反馈式调优(如 \textbf{GRPO/RFT}),并以“格式/类型/坐标/内容”四粒度奖励,\textbf{收紧输出空间}同时提升细粒度正确性\cite{ui-venus}。作为收敛视角,近期综述将上述组件归并为\textbf{记忆—推理—I/O}三件套:\textbf{planning+reflection} 对齐 \textbf{reasoning},\textbf{skills/macros} 属于 \textbf{I/O} 的动作产出形态,\textbf{memory} 独立,便于在实现与评测时统一术语与接口\cite{llm-ga}。

训练与数据侧的配套通常与\textbf{结构化动作}相匹配:通过 \textbf{RL-finetune} 与反馈式调优(如 \textbf{GRPO/RFT}),并以“格式/类型/坐标/内容”四粒度奖励,\textbf{收紧输出空间}同时提升细粒度正确性,从而在“反思—记忆—技能”的外圈外,再加一层\textbf{可学习的约束}\cite{ui-venus}。此外,接口位形也影响模块产物:\textbf{VLM} 倾向输出文本/JSON,经由映射进入动作空间;\textbf{VLA} 则直接产出动作向量/分布,二者在无效率、校准与 OOD 行为上的取舍需要在同一协议中对照评测\cite{Benchmarking-VLA-VLM}。\textit{要让这些模块更稳/更强,下一步问题就是:如何学习与调优。}

% ==========
% 2.4 学习范式
% ==========
\subsection{学习范式(Learning Paradigms)}
% 作用:集中说明“怎么学/怎么调”,与 2.3 的模块和 2.2 的协议形成闭环;突出指令化(MGI)、RL-finetune 与工具增强的互补性。
% 承上启下:本节结尾过渡到 2.5——即便如此,系统性短板依然存在,需在评测中直面。

指令化与条件化提供了“看懂—再行动”的可解释路径:\textbf{R2-Play} 将多模态游戏指令(\textbf{multimodal game instructions, MGI})并入 \textbf{Decision Transformer},用“\textbf{游戏描述—轨迹—操作引导(含关键元素位置)}”三段式模板共享跨任务知识,在多任务与泛化上报告相对优势,可作为\textbf{提示/指导结构}的模板参考\cite{r2-play}。对结构化动作任务而言,\textbf{RL-finetune} 配合四粒度奖励把“合法性+定位/文本输入”压进可学习信号,\textbf{直接对齐} 2.2 节中的“无效动作治理”目标\cite{ui-venus}。

为在资源与稳态间取舍,\textbf{工具增强型 MLLM} 的综述总结了外部工具(OCR/检索/计算/专家模型等)在 \textbf{tool use / MCP-style} 编排下对延迟与鲁棒性的影响与边界,提示通过检索与轻量工具链分担大模型负载;而接口对照研究将 \textbf{VLM+映射} 与 \textbf{VLA直出动作}并置,指出两类路径在无效率、置信校准与 OOD 表现上的差异,需要结合具体协议与数据分布权衡\cite{tool-aug-mllm,Benchmarking-VLA-VLM}。\textit{即便“能做—能比—能学”已形成闭环,模型仍有系统性短板,必须在评测中直面。} 

% —— 收束句,回钩全文母题 ——
% \noindent\textit{综上,文献给出的“\textbf{可行}—\textbf{可比}—\textbf{可稳}—\textbf{可学}—\textbf{有边界}”五段证据限定了本文后续系统与评测的口径:以 \textbf{GCC} 为统一接口,以 \textbf{结构化输出 + 合法动作映射} 降无效;以 \textbf{可复现协议与脚手架} 做对照;并用 \textbf{OAS/RT/APO} 捕捉伴随式实时体验。}