\section{Literature Review}
% > 说明:按taxonomy组织;每小节末有“与本文关系”。Cradle放在2.2(GUI/GCC)。涉及了2.1-2.9。

% ==========
% 2.1
% ==========
\subsection{Perception: Modalities \& Grounding}
% > 输入模态与定位;本节仅界定输入模态与\emph{grounding} 能力的术语背景。不涉及实现选择。涉及2.3-2.4。
该方向通常以屏幕帧(screen frames)或短视频(video clips)作为主要输入,辅以窗口/坐标等轻量上下文;可选接入音频(audio)以形成语音闭环(ASR/TTS)。多模态模型在此承担检测/描述/定位(detection/description/grounding)、UI元素识别与状态读出等能力;与直接产出动作的\emph{VLA} 相比,\emph{VLM+tool} 将视觉理解与动作执行解耦,通过\emph{结构化调用(structured tool calls)}或\emph{技能库(skills/macros)}完成闭环。文献中的代表实践包括:\emph{UI-Venus} 在\emph{screenshot-only} 条件下通过\emph{结构化动作}实现端到端导航\cite{ui-venus};\emph{V\!-MAGE} 强调\emph{visual-only/continuous-space} 设定下的定位、时机与视觉记忆压力\cite{v-mage};\emph{lmgame} 提供感知/记忆脚手架(scaffolds)以稳定交互与提示方差\cite{lmgame}。

% > 衔接2.1的术语与范围背景,给出Agent AI的工作定义与grounding取向;涉及该文的导论与概念部分。
\paragraph{Agent AI(multimodal interaction survey)}
该综述给出“\emph{Agent AI}”的工作定义:能感知多模态输入(视觉/语言/环境信号)并在具身或虚拟环境中产生动作(embodied actions)的交互系统;作者从“下一步具身动作预测(next-embodied action prediction)”出发,讨论外部知识(external knowledge)、多传感输入(multi-sensory inputs)与人类反馈(human feedback)在\emph{grounded} 场景下提升稳健性的作用,并主张通过虚拟/模拟环境加速研究进展\cite{agent-ai}。与本文关系:作为2.1小节的术语锚点与范围图谱,用于对齐“感知—定位—交互”相关表述。


% ==========
% 2.2
% ==========
\subsection{Action Interfaces: GUI (GCC) \& MCP-style Orchestration}
% > 说明:以GUI(GCC)作为统一执行通道;MCP作为内部“技能总线(skill bus)”的注册/编排思想(不讨论API适配)。Cradle在此作GUI可行性的代表。涉及2.6-2.7。

在动作接口上,\textbf{GUI} 路线以 \textbf{General Computer Control (GCC)} 为统一通道(\emph{screen-in, keyboard/mouse-out}),强调跨应用/跨游戏的可迁移性(portability)与人类同态交互(human-homomorphic interface)。代表性工作 \emph{CRADLE} 展示了在不依赖应用专用接口的前提下,通过\emph{规划—技能整理(skill curation/registry)—反思—记忆}的管线完成长链路任务(desktop/games),为 \emph{GUI} 可行性提供了实证支持\cite{CRADLE}。与此同时,\textbf{MCP}(Model Context Protocol)提供了\emph{模块注册/编排(module registration/orchestration)}的协议化思路:在不改变输出仍为 \emph{GUI} 的条件下,\emph{skills/macros、planning、memory、reflection} 等可在统一接口下组织,便于可复现实验与消融比较\cite{ORAK}。

\paragraph{Takeaway}
文献显示:\emph{GUI(GCC)}提供统一接口与较低移植门槛;协议化编排(如\emph{MCP})有助于模块化与复现性。与本文关系:本节作为动作接口背景与术语界定。

如 \autoref{fig:cradle-overview} 所示,CRADLE 以统一的 GUI(GCC)通道展示了从“屏幕输入→内在推理→键鼠控制”的闭环。

\picHere{./assets/images/from-papers/cradle01.jpg}{0.9\linewidth}
{An overview of the CRADLE framework: CRADLE takes video from the computer screen as input and outputs computer keyboard and mouse control determined through inner reasoning (planning, skill curation, reflection, memory) \cite{CRADLE}.}
{fig:cradle-overview}

\paragraph{UI-Venus(screenshot-only)}
端到端 GUI 导航,无需 planner/A11y;强调\emph{截图$\to$结构化动作}的通道在真实平台可达到具有竞争力的结果(如 AndroidWorld 的 pass@1 与 ScreenSpot 系列定位)\cite{ui-venus}。与本文关系:作为 screenshot-only 路线可行性的文献实例。

\paragraph{LLM-brained GUI agents(survey)}
该综述将“以 LLM 为中枢的 GUI 智能体”作为统一对象,围绕 GUI 自动化的\emph{框架}、\emph{数据与训练}、面向动作的\emph{专门化模型(large action models, LAM)}与\emph{评测基准/指标}展开,总结跨 Web/移动/桌面平台的通用交互能力与挑战,并给出未来路线图\cite{llm-brained-gui}。
% 与本文关系:作为 2.2 小节的范围总览与术语对齐,亦为 2.5/2.6 的基准与指标综述提供参照。

\paragraph{(M)LLM-based GUI agents(survey)}
该综述将 GUI 智能体拆解为四大组件:\emph{perception}(文本解析与多模态理解)、\emph{exploration/knowledge}(内部模型、历史回放与外部检索)、\emph{planning}(推理与任务分解)与 \emph{interaction}(动作生成与安全控制),并回顾跨桌面/移动/Web 的研究进展;作者据此指出元素定位、知识检索、长时规划与安全执行控制仍是前沿挑战,同时强调现有评测在指标与协议上亟需标准化\cite{mllm-gui-survey}。
% 与本文关系:作为 2.2 的“体系/术语锚点”,并为 2.5/2.6 的基准与指标对齐提供参考。

\paragraph{OS Agents(OS-level scope survey)}
该综述在 OS-level 上界定了 (M)LLM 驱动的计算设备智能体:以操作系统提供的接口(如 GUI/CLI)为通道,跨电脑/手机/浏览器执行任务;作者提出“环境—观测空间—动作空间”的要素划分,并将“理解/规划/动作落地(grounding)”作为核心能力,系统回顾领域化基础模型、代理框架、评测指标与基准,以及产业产品版图与安全威胁\cite{os-agents}。
与本文关系:提供 2.2 的术语与范围锚点,可与 \emph{CRADLE/UI-Venus/ORAK} 等文献在“GUI(GCC)/评测与安全”维度对齐。

% ==========
% 2.3
% ==========
\subsection{Agentic Modules: Planning, Memory, Reflection, Skills}
% > 机制视角综述;仅作分类与代表做法回顾。涉及2.1, 2.2, 2.4。
\todo{规划(planning)、记忆(memory, 用户偏好/历史)、反思(self-reflection, 纠错/风格一致)、技能库(skills/macros, 原子→复合)。
与本文关系:仅作机制分类与代表性做法的回顾。}

文献中的 agentic modules 常见于四类:\emph{planning}(分解与策略选择)、\emph{memory}(短长时/用户偏好)、\emph{self-reflection}(纠错与风格一致)与\emph{skills/macros}(原子→复合操作)。例如 \emph{CRADLE} 组合了\emph{planning/skills/reflection/memory} 的管线以缓解长链路误差累积,\emph{ORAK} 在统一评测中对上述模块进行消融比较\cite{CRADLE,ORAK};\emph{UI-Venus} 则在训练与数据层面探索\emph{轨迹历史对齐}与\emph{稀疏动作增强}\cite{uivenus_rft}。
与本文关系:作为模块分类与代表做法的回顾。

\paragraph{历史对齐与稀疏动作增强}
提出 \emph{Self-Evolving Trajectory History Alignment \& Sparse Action Enhancement}:
用当前模型重写历史“思维—动作”轨迹以对齐风格/细节,并上采样稀疏但关键动作(如 LongPress),以改善长链路一致性与泛化\cite{uivenus_rft}。与本文关系:作为处理长链路与长尾动作的文献做法。

% > 作为模块讨论的背景补充,点出“下一步动作预测/HF”与agentic能力的联系。
补充而言,\cite{agent-ai} 将“下一步具身动作预测(next-embodied action prediction)”与人类反馈(human feedback)并置为提升\emph{agentic} 能力的关键因素,强调在\emph{grounded} 环境中校准策略与记忆的重要性。与本文关系:用于模块分类讨论的语义背景与术语对齐。


% ==========
% 2.4
% ==========
\subsection{Learning Paradigms: Zero-shot, RAG, Finetune, IL/RL, Distillation}
% > 训练与推理范式。涉及2.1, 2.3。
\todo{零样本/提示工程、检索增强(RAG for UI schema/FAQ)、轻量微调(LoRA)、模仿/强化(IL/RL)、蒸馏到小模型。与本文关系:范式综述,不含实现承诺。}

以“\emph{指令化}(instructionalization)”增强 RL 代理的上下文理解是一条代表性路线。\emph{R2-Play} 将多模态游戏指令(MGI)并入 \emph{Decision Transformer}(\emph{DTGI}),并通过超网络(\emph{SHyperGenerator})在训练任务与未见任务间共享知识;作者报告多模态指令较文本/轨迹单模态在多任务与泛化上更优(动机见 \autoref{fig:r2play-motivation}),MGI 的三段式结构——\emph{game description}、\emph{game trajectory}、\emph{game guidance}(含动作、语言引导及关键元素位置)——给出了指令模板(见 \autoref{fig:r2play-mgi})\cite{r2-play}。与本文关系:作为“指令化/Decision Transformer”方向的文献背景。

\picHere{./assets/images/from-papers/r2-play01.jpg}{0.7\linewidth}
{Imagine an agent learning to play Palworld. (1) The agent exhibits confusion when only relying on textual guidance. (2) The agent is confused when presented with images of a Pal sphere and a Pal. (3) The agent understands how to catch a pet through \emph{multimodal guidance}, which combines textual guidance with images of the Pal sphere and Pal \cite{r2-play}.}
{fig:r2play-motivation}

\picHere{./assets/images/from-papers/r2-play02.jpg}{0.9\linewidth}
{An illustrative example of \emph{multimodal game instructions (MGI)}. Each instruction consists of three sections: \emph{game description}, \emph{game trajectory}, and \emph{game guidance} (including action, language guidance, and the position of key elements) \cite{r2-play}.}
{fig:r2play-mgi}

\paragraph{RFT(GRPO)与动作粒度奖励}
将奖励拆分为\emph{格式/动作类型/坐标/内容}四部分并加权,以同时度量\emph{结构化输出合规性}与\emph{细粒度定位/文本输入}正确性,作为 GUI 导航中 RL-finetune 的代表做法之一\cite{uivenus_rft}。与本文关系:作为 RL-finetune 在 GUI 导航中的奖励设计范例。

\paragraph{Tool-augmented MLLMs(survey)}
该综述在 MLLM 框架下扩展“工具”概念,覆盖 API、专家模型与知识库等外部手段,并从“数据—任务—评测”三条主线梳理近年的方法学:在任务层面,总结了外部工具在六类挑战场景(多模态RAG、推理、幻觉、 安全、代理、视频感知)中的典型用法;在流程层面,以 MRAG 的“检索—重排—整合”三段式为例展示工具化管线;在评测层面,指出既有指标难以全面刻画多模态生成与对齐,主张引入工具协助的更系统评测\cite{tool-aug-mllm}。
% 与本文关系:作为 2.4 的“工具/RAG 与学习范式”术语与方法学背景,并为 2.6(评测协议)与 2.8(攻防/安全)提供参照。

% ==========
% 2.5
% ==========
\subsection{Benchmarks \& Datasets (OS-like, Games, Desktop)}
% > 三段法——Orak(统一评测/消融/MCP思想)→ lmgame-Bench(脚手架+污染控制)→ Procedural-generation(OOD方法学)。涉及2.6。每段末“与本文关系”一句,不作采用承诺。

\paragraph{Orak(统一评测/消融/MCP思想)}
\emph{ORAK} 通过 \emph{MCP} 实现\emph{plug-and-play} 的代理—环境解耦,并在统一配置下检验
\emph{planning / reflection / memory / skills} 等\emph{agentic modules} 的边际贡献(ablation),配套
\emph{Leaderboard/Battle Arena} 与训练轨迹数据(fine-tuning trajectories),将\emph{机制—性能—配置}一体化呈现\cite{ORAK}。
与本文关系:作为“统一评测与消融”的代表性基准。

\paragraph{Procedural-generation(OOD方法学)}
基于\emph{procedural generation} 的开放式评测在可控生成下构造\emph{OOD} 与多步任务压力,比较
\emph{VLA/VLM} 在\emph{架构/训练数据/输出后处理}等变量下的泛化与稳健性,并配套工具链以保证\emph{reproducibility}\cite{Benchmarking-VLA-VLM}。
与本文关系:作为 OOD/变量可控的评测方法学背景。

\paragraph{lmgame-Bench(脚手架与污染控制)}
\emph{lmgame} 将“\emph{游戏→评测}”系统化:用 \emph{Gym-style} 接口与\emph{perception/memory scaffolds}
稳定\emph{prompt} 并剔除\emph{污染},在多模型下获得良好分离度,并通过\emph{相关性分析}展示“各游戏探测的能力混合不相同”;另报告单一游戏的 \emph{RL} 训练对\emph{未见游戏}/\emph{外部规划任务}存在迁移\cite{lmgame}。
与本文关系:作为“脚手架/污染控制/迁移观察”的评测文献。

…为减少提示方差并抑制污染,lmgame 以模块化脚手架稳定“感知—记忆—推理”的交互回路(见 \autoref{fig:lmgame-harness})。

\picHere{./assets/images/from-papers/lmgame01.jpg}{0.9\linewidth}
{lmgame-Bench uses modular harnesses—such as perception, memory, and reasoning modules—to systematically extend a model’s game-playing capabilities, enabling iterative interaction loops with a simulated game environment \cite{lmgame}.}
{fig:lmgame-harness}

\paragraph{V\!-MAGE(vision-centric, visual-only, continuous-space)}
该框架以\textbf{仅视觉输入}与连续空间的游戏环境,评测多模态模型的视觉中心能力,覆盖定位、轨迹追踪、时机、视觉记忆及更高层时序推理;其评测管线支持分离“模型/策略”,并采用 Elo 风格排名进行相对强度比较;作者报告现有模型与人类表现存在差距、常见感知错误与锚定偏差,且有限历史上下文会限制长时规划\cite{v-mage}。
与本文关系:作为视觉中心评测的代表性基准。

\picHere{./assets/images/from-papers/v-mage01.jpg}{0.9\linewidth}
{The overview of the V-MAGE benchmark, designed to evaluate vision-centric capabilities and higher-level reasoning of MLLMs across 5 free-form games with 30+ levels \cite{v-mage}.}
{fig:v-mage-overview}

% ==========
% 2.6
% ==========
\subsection{Evaluation Protocols \& Metrics}
% > 强关联本文贡献;此处仅回顾文献做法,不作实现承诺。

文献在\emph{客观指标}上常使用任务成功率(\emph{success/pass@k})、完成时间(\emph{time-to-completion})、误操作/回滚相关比率(\emph{misclick/rollback rate})以及延迟(\emph{latency},如语音往返 \emph{voice RTT})等;在\emph{主观或行为性指标}上,亦见用户采纳程度(\emph{advice adoption})与满意度等表述\cite{ORAK,lmgame,Benchmarking-VLA-VLM}。

此外,针对类别分布不均(class imbalance)的场景,亦有工作同时报告\emph{macro-averaged} 与\emph{micro} 指标,以减轻多数类偏置对总体判断的影响\cite{Benchmarking-VLA-VLM}。

在协议层面,部分工作强调\emph{结构化输出与约束解码(structured output \& constrained decoding)}以降低无效动作和坐标偏差;也有将\emph{无效动作率}纳入评估与惩罚设计的做法(例如将动作拆分为\emph{格式/类型/坐标/内容}等粒度进行评测与训练信号的组合)\cite{ui-venus}。与“游戏→可靠评测”相关的工作则强调记录\emph{后处理/解码策略(post-processing)}、\emph{提示方差}与\emph{污染}控制,以减少实现细节对可比性的影响\cite{lmgame}。

在跨任务比较上,亦有采用\emph{Elo 风格排名(Elo-style ranking)}报告相对强度的实践,便于处理关卡难度不均与多任务汇总\cite{v-mage}。同时,基于\emph{procedural generation} 的方法学将\emph{架构/数据/后处理}视作可控变量进行统一对比,强调\emph{OOD} 情形下的稳健性与可复现性\cite{Benchmarking-VLA-VLM}。

与本文关系:本小节仅提供指标与评测协议的文献背景与术语对齐,不涉及实现承诺。


% ==========
% 2.7
% ==========
\subsection{Deployment \& Real-time Considerations}
% > 工程现实回顾:资源/时延/交互链路;不给实现承诺。涉及2.2, 2.6。
部署相关文献关注资源与实时性约束:\emph{本地—云混合(local–cloud hybrid)}与\emph{推理量化(inference quantization, 如 INT4/FP8)}用于降低时延与成本;\emph{流式解码(streaming decoding)}与\emph{语音中断(barge-in)}用于缩短交互回路;并评估对帧率与 CPU/GPU 占用的影响。为保证可比性,协议层面强调固定提示、记录\emph{post-processing} 与环境版本;在真实设备/平台的在线评测也逐渐出现(如 AndroidWorld 场景与脚手架化交互)\cite{lmgame,ui-venus}。
与本文关系:作为工程现实与实时交互的背景回顾。


% ==========
% 2.8
% ==========
\subsection{Safety, Permissions \& Robustness}
% > 安全与鲁棒性回顾:权限范围、确认、影子执行、回滚/急停、审计。涉及2.2。
文献强调\emph{权限模型(permission scoping/whitelisting)}与\emph{操作确认(confirmation)}以约束高风险动作;\emph{影子模式(shadow execution)}先预测后执行以降低副作用,并配套\emph{回滚(rollback)}与\emph{急停(kill-switch)}保障可逆性与故障恢复。在 GUI 场景下,\emph{think–action mismatch} 揭示了多模态模型可能产生的“幻觉(hallucination)”与不一致风险,提示需要\emph{日志与审计(auditability)}以支持溯源与复查\cite{ui-venus}。
与本文关系:作为权限/鲁棒性与审计机制的文献背景。


如 \autoref{fig:uivenus_qual} 所示,think–action 不一致(mismatch)揭示了 MLLM 的“幻觉”(hallucination)风险\cite{ui-venus}。

\picHere{./assets/images/from-papers/ui-venus01.jpg}{0.95\linewidth}
{One trace of UI-Venus on the task named MarkorDeleteAllNotes in AndroidWorld. We can observe that UI-Venus successfully achieves the goal and has the reflection ability in Step 3. However, there also exists the conflict between think and action in Step 5, remaining as a future work about how to solve MLLM’s hallucination.\cite{ui-venus}}
{fig:uivenus_qual}

% ==========
% 2.9
% ==========
\subsection{Synthesis: Trends, Gaps \& Our Niche}
% > 综述收束到本文位置;仅做趋势与缺口陈述,末尾一句“与本文关系”定位。
当前趋势是在\emph{GUI(GCC)}通道上引入\emph{协议化/模块化}编排以支撑可复现实验与消融;真实多类型游戏的统一评测(如\emph{Orak})与\emph{visual-only/continuous-space} 的视觉中心评测(如\emph{V\!-MAGE})并行发展;把\emph{脚手架/污染控制}引入评测协议(如\emph{lmgame-Bench})成为常见做法。与本文关系:本文研究定位于\emph{伴随式(companion-style)}场景的文献回顾与术语/评测背景梳理。

% > 收束文献趋势,呼应grounded取向;不涉及实现承诺。
与基于 GUI(GCC)的实践并行,\emph{grounded} 的感知—行动—人类反馈闭环正成为多模态交互智能体的共识性趋势 \cite{agent-ai}.