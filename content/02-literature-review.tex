\section{Literature Review}
> 说明:采用taxonomy组织而非按时间;每小节末给1-2句“与本文关系”。Cradle放到2.2(GUI-only/GCC)而非开头。涉及了2.1-2.9。

\subsection{Perception: Modalities \& Grounding}
> 说明:输入模态与定位。涉及2.3-2.4。
\todo{视觉为主(screen/video)+可选音频(audio);VLM能力:检测/描述/grounding;提一嘴VLA(如果动作权重内生)与传统VLM+tool的差别。与本文:我们选轻量VLM,优先本地(on-device)与流式ASR/TTS。}

\subsection{Action Interfaces: GUI-only (GCC) vs API/MCP}
> 说明:动作接口对比的核心小节,放Cradle。涉及2.6-2.7。
\todo{定义GCC:screen-in, keyboard/mouse-out。代表作:\textbf{Cradle}(GUI-only,skill curation/registry, reflection/memory)。优点:通用性(generality)、可迁移性(portability);缺点:确定性/延迟。API/MCP:确定性高、速度快、但依赖适配。本文策略:GUI-first + API/MCP作为加速通道,GUI兜底。附一句对比表指引。}

\subsection{Agentic Modules: Planning, Memory, Reflection, Skills}
> 说明:机制视角。涉及2.1, 2.2, 2.4。
\todo{规划(planning)、记忆(memory, 用户偏好/历史)、反思(self-reflection, 纠错/风格一致)、技能库(skills/macros, 原子→复合)。说明这些机制如何提升长链路成功率与体验一致性。与本文:直接采纳skills+reflection+memory组合。}

\subsection{Learning Paradigms: Zero-shot, RAG, Finetune, IL/RL, Distillation}
> 说明:训练与推理范式。涉及2.1, 2.3。
\todo{列常见范式及成本/收益:零样本与提示工程、检索增强(RAG for UI schema/FAQ)、轻量微调(LoRA)、模仿/强化(IL/RL)、蒸馏到小模型。与本文:优先零样本+RAG,必要时小规模LoRA以稳UI。}

\subsection{Benchmarks \& Datasets (OS-like, Games, Desktop)}
> 说明:基准版图。涉及2.6。
\todo{按类型分:桌面/操作系统类(如OSWorld系)、游戏/模拟器类(如ALE等)与自建任务脚本。指出覆盖能力与缺口:缺少“伴随式建议/语音互动”的评测。与本文:定义我们的小型、可复现实验设置与演示脚本。}

\subsection{Evaluation Protocols \& Metrics}
> 说明:强烈关联本文贡献。涉及2.2, 2.7。
\todo{客观:success rate, time-to-completion, no-misclick/rollback rate, latency(voice RTT, frame→hint时间);主观:advice adoption, user satisfaction。与本文:将新增advice adoption与macro success作核心指标。}

\subsection{Deployment \& Real-time Considerations}
> 说明:工程现实。涉及2.2, 2.6。
\todo{本地/云混合、量化(INT4/FP8)、流式解码、语音中断(barge-in)、资源占用与帧率影响。与本文:给出延迟预算(如$\leq$ 500ms提示、$\leq$ 1.5s语音回路)。}

\subsection{Safety, Permissions \& Robustness}
> 说明:安全边界。涉及2.2。
\todo{权限模型(whitelist, scope)、操作确认、影子模式(shadow mode)先预测后执行、回滚/急停。与本文:作为系统必要模块。}

\subsection{Synthesis: Trends, Gaps \& Our Niche}
> 说明:综述收束到本文位置。关联全篇。
\todo{趋势:GUI-only通用→API/MCP混合确定性;机制:从对话到任务化/技能化;缺口:缺少“伴随式建议+语音”的统一评测与低延迟实现。本文niche:针对实时游戏的companion-style助手,提供可复现小型协议与演示。}
