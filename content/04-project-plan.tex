\section{Project Plan}
\label{sec:project-plan}
This section details the execution and validation strategy for the proposed methodology. It outlines the project's core objectives, the final deliverables, a two-phased development and evaluation plan, the project timeline, and a risk analysis.

\subsection{Project Objectives}
\noindent This project is defined by two primary, high-level objectives:
\begin{enumerate}
    \item \textbf{Engineering Objective:} To design, build, and deploy a novel, real-time, unified AI companion prototype. This system will be built on a modular, \textbf{MCP-style} architecture that successfully integrates a functional ``Agent'' core and a proactive, ``Relational'' core.
    
    \item \textbf{Validation Objective:} To execute a reproducible, phased evaluation plan to first validate the system's functional competence in controlled environments, and then to prove the core hypothesis: that the unified agent provides a measurably superior user experience in complex, dynamic environments.
\end{enumerate}

\subsection{Expected Deliverables}
\noindent The final deliverables for this project will consist of four main components:
\begin{itemize}
    \item \textbf{The System Prototype:} A functional source code implementation of the unified AI companion, including the central Orchestrator, all sub-modules (Functional, Relational, Safety), the perception pipeline, and the Live2D frontend.
    \item \textbf{Documentation:} Detailed documentation of the source code, system design, and module interactions, providing an understanding of the system's architecture and implementation.
    \item \textbf{The Evaluation Suite:} All scripts, configuration files, and collected data used for system validation, including the functional test suite and the qualitative user study surveys and (anonymized) results.
    \item \textbf{The Video Demonstration:} A recorded demo video showcasing the prototype's key functional and relational capabilities, highlighting the system's performance and user interaction in real-time.
\end{itemize}

\subsection{Development Phases and Evaluation}
The project will be executed in two distinct phases to manage complexity and validate the system incrementally.

\paragraph{Phase 1: Functional Core Validation (Simple \& Logic-Based Games)}
The first phase will focus on building and validating the core technical stack (Perception, Orchestrator, Functional Core, Safety). This will be tested in controlled, low-rhythm, or logic-based games, such as \textbf{2048} as a unit test, or \textbf{Stardew Valley} as a more complex scenario, to isolate and debug the agent's competence. A test suite of standardized, functional tasks (e.g., ``Achieve the 1024 tile,'' ``Maps to the General Store,'' ``Plant 10 seeds'') will be created. \textbf{Success in this phase} will be measured using SOTA (State-of-the-Art) metrics from the literature, including a high \textbf{Task Success Rate (TSR)} and a low \textbf{\texttt{Invalid\%}} of actions.

\paragraph{Phase 2: Unified System Validation (Complex \& High-DOF Games)}
Once the core agent is functionally competent, the second phase will integrate the \textbf{Relational Core} (including the Live2D frontend) and transfer the full system to more complex, high-degree-of-freedom (DOF) environments like \textbf{Minecraft}. This will test the agent's ability to handle long-horizon planning and real-time threats. The core hypothesis will then be tested via an \textbf{A/B user study} with a small sample size (5-10 users)\footnote{An A/B user study involves comparing two or more versions of a product or feature to evaluate user preferences or performance. A sample size of 5-10 users is typically used in early-stage pilot studies to gather preliminary feedback before larger-scale testing.}. \textbf{Group A (Control)} will play with the \textbf{Functional Core only} (a ``silent tool''). \textbf{Group B (Test)} will play with the \textbf{Full Unified System}. Success will be measured using a qualitative survey to assess ``companionship,'' ``proactivity,'' and ``cognitive workload'' \cite{human-centered-eval, Cooperation-Player-AI}. A stretch goal for this phase, if time permits, is to test the OOD transferability of the agent to a highly complex RPG like \textit{Genshin Impact}.

\subsection{Project Timeline}
The project is scheduled for 18 active work units, from late November 2025 to mid-May 2026. The timeline is structured around three main development blocks: 4 units in Semester 1 (late Nov - Dec), \textbf{at least 4 units during the Winter Holiday} (Jan - Feb), and 10 units in Semester 2 (Mar - May). The development plan is not sequential but parallel, centered on a core ``brain'' (the Orchestrator) with other modules being developed, integrated, and refined concurrently. To manage this iterative process, modules are versioned (e.g., v1, v2): a `v1` module represents an initial, stubbed-out implementation, while a `v2` module represents a more refined, fully-integrated, and tested version.

\paragraph{Long-Running Development Tasks (Units 2-15)}
Several core components will be in development for the majority of the project. The \textbf{Orchestrator \& MCP Bus} will be the first component built (v1) and will be continuously updated (v2, v3) as new modules are plugged in. The \textbf{Functional Core (RFT/PORTAL)} and the \textbf{Perception Module (YOLO/OCR)} will be developed and trained for Phase 1 games (e.g., \textit{Stardew Valley}) during the holiday, and then undergo transfer-learning and refinement for the Phase 2 game (\textit{Minecraft}) in Semester 2.

\paragraph{Block 1: Core Framework (Units 1-4)}
The initial 4 units will be dedicated to building the ``brain''. This includes finalizing the \textbf{MCP API design} and implementing the initial \textbf{Orchestrator (v1)}. In parallel, a simple \textbf{Live2D Frontend (v1 stub)} and the \textbf{Voice Loop Module} will be connected to validate the core relational message-passing loop.

\paragraph{Block 2: Functional Core (Units 6-9)}
The 4-unit Winter Holiday block will be focused on the first major bottleneck: the ``eyes'' and ``hands''. The \textbf{Perception Module (v1)} for \textit{Stardew Valley} and the \textbf{GCC I/O Module} will be built. This block is primarily dedicated to the intensive \textbf{RFT training} (or `PORTAL` BT-generation) of the \textbf{Functional Core (v1)}.

\paragraph{Block 3: Integration \& Validation (Units 11-20)}
The final 10-unit block is for integration, validation, and finalization. The \textbf{Safety Core} (inspired by `BacktrackAgent`) will be built and plugged in. The \textbf{Phase 1 (Functional) Evaluation} will be run. Concurrently, the \textbf{Perception (v2)} and \textbf{Functional (v2)} modules will be adapted for the high-DOF \textit{Minecraft} environment. The \textbf{Relational Core (v2)} will be fully integrated with its proactive triggers. The final weeks will be dedicated to running the \textbf{Phase 2 (A/B User Study)}, followed by final code cleanup, documentation, and demo video preparation.

\picHere{assets/images/gantt_chart.png}{1.0\textwidth}{
\todo{A detailed weekly Gantt chart visualizing these parallel tasks will be developed.}}{fig:gantt-chart}

\subsection{Risk and Ethics Analysis}
This plan identifies three primary technical risks and one ethical consideration:

\paragraph{Risk 1: Latency} Real-time LLM inference (>100~ms) in Phase 2's combat scenarios will break immersion \cite{liu2023llm}. The mitigation is architectural: the system will use a hybrid model inspired by \texttt{PORTAL}, where high-frequency actions (like combat dodges) are handled by a zero-latency compiled Behavior Tree, reserving the LLM for high-level planning \cite{PORTAL}.

\paragraph{Risk 2: Safety \& Error Accumulation} The agent may get stuck in loops or fail long-horizon tasks, especially in complex 3D environments. The mitigation is a dedicated Safety Core, inspired by \texttt{BacktrackAgent}, which includes a ``Verifier'' and ``Judger'' to detect and recover from semantic errors (e.g., ``player is stuck,'' ``mob approaching''), triggering a replan \cite{BacktrackAgent}.

\paragraph{Risk 3: OOD Generalization} The agent may fail to transfer from Phase 1 (e.g., \textbf{Stardew Valley}) to Phase 2 (e.g., \textbf{Minecraft} or \textit{Genshin Impact}), a known SOTA challenge \cite{Benchmarking-VLA-VLM}. This transfer is a core part of the research, and the risk is mitigated by using a game-agnostic \texttt{GCC} interface and a \texttt{PORTAL}-style policy generator, which has shown SOTA cross-game generalization \cite{PORTAL, CRADLE}.

\paragraph{Ethics: Privacy} The system requires capturing the user's screen and voice. The mitigation is to process all data \textbf{locally on-device}. No PII (Personally Identifiable Information) will be stored or transmitted.