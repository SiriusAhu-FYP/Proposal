\section{Introduction}
% Why worthy? (Passion, market, need)
% Demand -> Neuro-sama (success on market and tech) -> Timely: recent research support -> Project Definition

The demand for \textbf{interactive and companion-like} experiences in real-time entertainment is on the rise. Players are no longer satisfied with simple automation or static overlays; they seek dynamic, engaging partners that offer meaningful interaction. As gaming experiences evolve, the focus is shifting towards assistants that not only provide practical help but also enrich the player’s journey through companionship and engagement.

Games like AI2U: ``With You 'Til The End'' are already capitalizing on the demand for interactive experiences, offering players the novelty of engaging with generative AI characters \cite{ai2u_game}. At the same time, major companies such as NVIDIA, with its Avatar Cloud Engine (ACE), and Ubisoft, with its ``NEO NPCs'', are advancing foundational technologies to create autonomous agents that enhance gameplay by offering more than just conversation~\cite{nvidia_ace_autonomous_2025, ubisoft_neo_npc_2024}. These developments demonstrate the growing commercial viability of AI-driven experiences, supported by increased player engagement and media attention~\cite{inworld_market_validation_2023}.

The rise of AI-driven virtual streamers, especially the Neuro-sama phenomenon, highlights a significant shift in both technology and community-driven commercialization. Neuro-sama, an AI-powered VTuber, engages in real-time conversations and dynamic gameplay, capturing the attention of a wide audience \cite{neurosama_youtube, streamelements_state_2024, streamscharts_q4_2024_landscape}. Although Neuro-sama remains closed-source, its success has sparked a vibrant open-source ecosystem, with developers working to replicate or expand upon its capabilities \cite{open_llm_vtuber, airi_project, kimjammer_neuro}. This technological shift has been met with strong commercial traction on platforms like Twitch, underscoring the growing market demand for AI that combines both utility and companionship \cite{neurosama_hype_streamscharts_2025}.

Building on the convergence of market demand and emerging technological feasibility, this project aims to develop a prototype for a companion-style assistant. The assistant will act as a persistent, in-game partner that uses a \textbf{full voice-loop} to provide \textbf{event spotting} and \textbf{tactical guidance}, offering real-time support without interrupting gameplay. The goal is to enhance the player’s experience through dynamic, interactive assistance. The technical scope of the project involves the following three key techniques:

\begin{enumerate}
    \item \textbf{Unified Action Interface via GUI (GCC)}: The system will use a human-homomorphic interface with a screen-in, keyboard/mouse-out paradigm. This eliminates the need for platform-specific interfaces or game-specific programming APIs, ensuring cross-platform adaptability without requiring specialized API integration \cite{CRADLE, ui-venus}.

    \item \textbf{Constrained Action Generation with Structured Output}: Actions will be selected from a predefined set of valid actions (e.g., ``move forward'', ``open inventory'') and formatted using structured output (e.g., JSON with specified field). This method reduces errors like hallucinations and ensures that actions are legal, predictable, and reproducible in real-time \cite{Benchmarking-VLA-VLM}.

    \item \textbf{Low-Coupling Orchestration}: The system will employ an MCP-style orchestration model that supports modular, plug-and-play components. This architecture enables easy integration of new skills or modules, ensuring scalability and flexibility. It also allows for future updates and system adjustments without significant changes to the core structure \cite{ORAK}.
\end{enumerate}

With a clear understanding of market demand and technological feasibility, this project aims to develop a companion-style assistant that seamlessly integrates into gameplay, offering both utility and companionship to enhance the overall player experience.