\section{Introduction}
% Why worthy? (Passion, market, need)
% Demand of interactive & companion experiences (Market possibility) -> Game aspect support: Indie games & Big companies -> Stream / Tech(? not sure) aspect support: Neuro-sama and related open-source community (success on market and tech) -> Vision of Project -> Tech scope -> Summary

The demand for \textbf{interactive and companion experiences} in real-time entertainment is on the rise. Players are no longer satisfied with simple automation or static overlays; they seek dynamic, engaging partners that offer meaningful interaction. As gaming experiences evolve, the focus is shifting towards assistants that not only provide practical help but also enrich the player’s journey through companionship and engagement.

Games like AI2U: ``With You 'Til The End'' are already capitalizing on the demand for interactive experiences, offering players the novelty of engaging with generative AI characters \cite{ai2u_game}. At the same time, major companies such as NVIDIA, with its Avatar Cloud Engine (ACE), and Ubisoft, with its ``NEO NPCs'', are advancing foundational technologies to create autonomous agents that enhance gameplay by offering more than just conversation~\cite{nvidia_ace_autonomous_2025, ubisoft_neo_npc_2024}. These developments demonstrate the growing commercial viability of AI-driven experiences, supported by increased player engagement and media attention~\cite{inworld_market_validation_2023}.

The rise of AI-driven virtual streamers, especially the Neuro-sama phenomenon, highlights a significant shift in both technology and community-driven commercialization. Neuro-sama, an AI-powered VTuber, engages in real-time conversations and dynamic gameplay, capturing the attention of a wide audience \cite{neurosama_youtube, streamelements_state_2024, streamscharts_q4_2024_landscape}. Although Neuro-sama remains closed-source, its success has sparked a vibrant open-source ecosystem, with developers working to replicate or expand upon its capabilities \cite{open_llm_vtuber, airi_project, kimjammer_neuro}. This technological shift has been met with strong commercial traction on platforms like Twitch, underscoring the growing market demand for AI that combines both utility and companionship \cite{neurosama_hype_streamscharts_2025}.

The success of these AI-driven personalities validates the market's desire for this combination of utility and companionship; this project builds upon this principle, seeking to translate this proven engagement from a broadcast-entertainment model into a player-centric, companion-style assistant. This assistant is designed to augment, not automate, the player's agency. Its role is distinct from broadcast-focused AI-Vtubers and player-replacing automation bots; it is scoped as a persistent, in-game partner that uses a \textbf{full voice-loop} to provide support that is both relational and functional. The system's core capabilities are threefold: offering \textbf{proactive, contextual companionship} by using in-character awareness to identify opportunities (e.g., spotting missed items); engaging in \textbf{collaborative problem-solving} to understand player goals and access external knowledge (e.g., researching online strategies); and supporting \textbf{on-demand task delegation}, allowing the player to hand over control for specific, well-defined tasks (e.g., 'explore this area while I'm away').
% After reading ⬆️, reader's first question: "Oh yeah? That sounds hard. How?"

The technical scope to achieve these capabilities involves three key techniques. First, the system will use \textbf{a Unified Action Interface via GUI}, a human-homomorphic interface with a screen-in, keyboard/mouse-out paradigm. This approach eliminates the need for game-specific APIs and ensures cross-platform adaptability \cite{CRADLE, ui-venus}. Second, actions will be managed through \textbf{Constrained Action Generation with Structured Output}. This method, which formats commands from a predefined set of valid actions (e.g., 'move forward') into structured output (e.g., JSON), reduces errors like hallucinations and ensures that delegated tasks are legal and reproducible \cite{Benchmarking-VLA-VLM}. Third, the system will employ \textbf{a Low-Coupling Orchestration} (i.e., modularization by MCP-style). This plug-and-play approach is crucial, ensuring both scalability for integrating new skills like problem-solving and task-delegation, and testability which enables systematic ablation studies to evaluate each component's contribution \cite{ORAK}.

With a clear understanding of market demand and technological feasibility, this project aims to develop a companion-style assistant that seamlessly integrates into gameplay. By demonstrating a system capable of this deeper, functional partnership—blending both companionship and shared agency—this work will enhance the overall player experience through dynamic, interactive assistance.