\section{Introduction}
> 说明:交代问题空间(游戏伴随式助手,companion-style game agent)、范围(scope)、研究缺口(gap)、本文立场与贡献(positioning \& contributions)。涉及了2.1-2.4和2.6-2.7。

\subsection{Problem Setting \& Motivation}
\todo{背景:实时游戏场景(real-time gaming),玩家需要事件提示(event spotting)、策略建议(tactical guidance)、低延迟语音交互(voice loop)。动机:现有VLM/VLA在桌面/游戏的落地与稳定性存在鸿沟。}

近年来,面向玩家的智能交互快速涌现:从\emph{AI游戏主播/虚拟角色}到\emph{LLM驱动的NPC/插件},社区与产业侧案例表明``大模型+游戏交互''具备显著关注度与潜在影响(game changer potential)。然而,这些案例多为定制工程,缺乏统一接口与可复现实验协议。本文聚焦\emph{伴随式(companion-style)}实时助手,围绕\emph{统一动作接口}与\emph{低延迟体验}提供可复现的方法与评测。

\paragraph{Industry/Community Signals}

除学术工作外,社区与产业侧的``AI$\times$ 游戏/直播''案例为本研究提供了现实动机。例如 \emph{Neuro-sama} 及其开源复刻框架\cite{neurosama_youtube,open_llm_vtuber,kimjammer_neuro,airi_project},以及叙事解谜作品 \emph{AI2U: With You 'Til The End}\cite{ai2u_game} 展示了``对话即操作''(dialogue-as-action)与高交互度(LLM-controlled NPCs)的设计可能性。我们据此聚焦于``游戏 + 大模型交互''的可复现路径,但在本文中\emph{不将这些案例视为方法有效性的学术证据},而是将研究问题落在\emph{统一动作接口}(action interface:GUI-only/GCC vs.\ API/MCP)、\emph{低延迟体验}(low-latency)与\emph{评测协议}(evaluation protocols)上。



\subsection{Scope \& Working Definitions}
\todo{给出工作定义(working definitions):多模态(multimodal)、动作接口(action interface: GUI-only vs API/MCP),伴随式助手(companion-style),短时托管(autopilot/macros),评测术语(success rate, latency, advice adoption)。}

我们将\textbf{动作接口}(action interface)划分为\emph{GUI-only}与\emph{API/MCP}两类。

\textbf{GUI-only}以\textbf{General Computer Control (GCC)}为代表:\emph{screen-in, keyboard/mouse-out}的人类同态接口(human-homomorphic interface)。代表作\emph{Cradle}展示了在不依赖应用API的前提下完成长链路桌面/游戏任务的可行性;本文在此基础上采取\emph{GUI-first}并\emph{机会主义接入API/MCP}以提升确定性与效率(determinism \& efficiency)。\cite{tan2024cradle}

% Intro: Scope & Working Definitions / 或 Evaluation Preview 末尾一段(3句以内)
作为与本文评测设置相关的代表性工作,\emph{Orak} 提供了覆盖 12 款真实电子游戏、跨 6 大类型的训练与评测基准,并以 \emph{Model Context Protocol (MCP)} 实现\emph{plug-and-play} 的代理与环境对接;其\emph{Leaderboard/Battle Arena}与\emph{agentic modules}消融,为比较不同模块与输入模态(text/vision)提供了统一框架. \cite{park2025orak}我们在本文中参考其“统一评测—模块消融—可复现配置”的思路,结合本场景的实时性需求构建更贴合“伴随式助手(companion-style)”的评测协议。

作为开放式环境下的系统评测参考,我们采用基于\emph{procedural generation}的统一框架来度量\emph{VLA/VLM}在多步轨迹与OOD设定中的表现,并将\emph{架构/数据/输出后处理}作为可控变量纳入对比\cite{guruprasad2025benchmarkingvisionlanguage}. 


\subsection{Key Challenges}
\todo{长链路稳定性(long-horizon stability)、UI变化鲁棒(robustness)、延迟预算(latency budget)、权限安全(permissions/rollback)、跨游戏迁移(generality/portability)。}

\subsection{Our Positioning \& Contributions}
\todo{工程立场:GUI-first + 机会主义API/MCP;引入skills/macros、planning/memory/reflection、语音链路;提出面向伴随式助手的评测协议(advice adoption, voice RTT, macro success)。可列1-3条要点。}

\subsection{Design Principles \& System Preview}
\todo{一句话系统图预告:screen/audio→VLM→LLM/agentic modules→(GUI kb/mouse | API/MCP)→safety guard(permissions, rollback, kill-switch)。把详图留到方法章节。}

\subsection{Summary of Findings (Optional)}
\todo{一句话总结文献趋势:从GUI-only通用性到API/MCP确定性,从对话式感知到任务化(taskification)与技能化(skills)。可选,若版面紧张可删。}
