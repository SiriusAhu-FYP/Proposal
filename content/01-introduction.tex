\section{Introduction}
> 说明:交代问题空间(游戏伴随式助手,companion-style game agent)、范围(scope)、研究缺口(gap)、本文立场与贡献(positioning \& contributions)。涉及了2.1-2.4和2.6-2.7。

\subsection{Problem Setting \& Motivation}
\todo{背景:实时游戏场景(real-time gaming),玩家需要事件提示(event spotting)、策略建议(tactical guidance)、低延迟语音交互(voice loop)。动机:现有VLM/VLA在桌面/游戏的落地与稳定性存在鸿沟。}

\subsection{Scope \& Working Definitions}
\todo{给出工作定义(working definitions):多模态(multimodal)、动作接口(action interface: GUI-only vs API/MCP),伴随式助手(companion-style),短时托管(autopilot/macros),评测术语(success rate, latency, advice adoption)。}

\subsection{Key Challenges}
\todo{长链路稳定性(long-horizon stability)、UI变化鲁棒(robustness)、延迟预算(latency budget)、权限安全(permissions/rollback)、跨游戏迁移(generality/portability)。}

\subsection{Our Positioning \& Contributions}
\todo{工程立场:GUI-first + 机会主义API/MCP;引入skills/macros、planning/memory/reflection、语音链路;提出面向伴随式助手的评测协议(advice adoption, voice RTT, macro success)。可列1-3条要点。}

\subsection{Design Principles \& System Preview}
\todo{一句话系统图预告:screen/audio→VLM→LLM/agentic modules→(GUI kb/mouse | API/MCP)→safety guard(permissions, rollback, kill-switch)。把详图留到方法章节。}

\subsection{Summary of Findings (Optional)}
\todo{一句话总结文献趋势:从GUI-only通用性到API/MCP确定性,从对话式感知到任务化(taskification)与技能化(skills)。可选,若版面紧张可删。}
