% ===== 段内小节(run-in)小宏:标题与正文同一行 =====
\newcommand{\phead}[1]{\par\smallskip\noindent\textbf{#1}\quad}
% 若想让小标题独占一行,则改为:
% \newcommand{\phead}[1]{\par\smallskip\noindent\textbf{#1}\par\smallskip}

\section{Introduction}
% > 作用:以“产品向 Proposal”口吻交代问题场景、范围(scope)、现有缺口(gap)与项目定位(positioning)。
% > 对标C.1模板的 1.1 Introduction/Background 与 1.2 Scope/Objectives;不在此承诺训练/参数等实现细节。

% ==========
% 1.1
% ==========
\subsection{Problem Setting \& Motivation}
% > 背景:实时游戏场景中的伴随式助手需求:事件提示(event spotting)、策略建议(tactical guidance)、低时延语音交互(voice loop)。
% > Gap:实践多为定制工程,缺少统一动作接口与可复现实验协议;研究侧在推进“统一评测/消融、污染控制/协议一致”等方法学。
本项目拟构建一个面向玩家的\emph{伴随式(companion-style)}实时游戏助手:围绕\emph{事件提示}、\emph{策略建议}与\emph{语音交互闭环}提供低时延(low-latency)体验与可复现评测(reproducible evaluation)。当前生态从\emph{AI游戏主播/虚拟角色}到\emph{LLM 驱动的 NPC/插件}快速涌现,但大量实践依赖场景定制,\emph{统一的动作接口}与\emph{可复现实验协议}仍不充分;研究侧正通过\emph{统一评测/模块消融}与\emph{污染控制/协议一致}等方法学收敛评测与对比\cite{ORAK,lmgame-bench}。本提案据此明确项目目标、范围与评测要素。

% ==========
% 1.2
% ==========
\subsection{Scope \& Working Definitions}
% > 目的:界定术语与范围(scope),使用“工作定义”语气;避免研究口吻与“仅xxx”表述。
% > 关键词:GUI(GCC), MCP-style orchestration, screenshot-first, structured output, evaluation protocol。
本项目\textbf{以 GUI(GCC)通道为主要执行路径}(\emph{screen-in, keyboard/mouse-out} 的人类同态接口 human-homomorphic interface),强调跨应用/游戏的通用性与可迁移性(portability)。代表性工作 \emph{CRADLE} 报告了在不依赖应用 API 的前提下,通过\emph{规划/技能整理/反思/记忆}的管线完成长链路任务的可行性\cite{CRADLE}。\,
\noindent\textbf{MCP-style 编排}在本项目中被用作\emph{内部模块/技能的注册与路由}(registration/orchestration)的通用思路;\textbf{不涵盖}针对具体应用/游戏的专用 API 适配工作。\,
关于“统一评测/消融、plug-and-play”的组织可参考 \emph{ORAK}\cite{ORAK};基于\emph{procedural generation} 的 OOD 方法学可参考 \emph{Benchmarking-VLA-VLM}\cite{Benchmarking-VLA-VLM};把真实游戏“转化为可靠评测”的协议化实践可参考 \emph{lmgame}\cite{lmgame-bench}。\,
在 GUI 场景中,\emph{UI-Venus} 展示了\emph{截图输入(screenshot)+ 结构化动作(structured output)}的端到端导航路径\cite{ui-venus};\emph{V\!-MAGE} 聚焦 \emph{visual-only/continuous-space} 的视觉中心评测\cite{v-mage}。相关工作将在第二部分系统梳理。

\phead{Screenshot-first GUI feasibility.}
在真实平台上,“截图输入 + 结构化动作输出”路线已有具体成绩:UI-Venus 在 AndroidWorld 报告 \textbf{65.9\%} \emph{pass@1},在 ScreenSpot-V2/Pro 分别为 \textbf{95.3\%}/\textbf{61.9\%}\,\cite{ui-venus},表明在不依赖 A11y/DOM 的条件下亦可实现可比导航能力。

\phead{MLLM architecture at a glance.}
图~\ref{fig:tool-aug-mllm01} 给出多模态大模型(MLLM)的通用结构:文本经 \emph{tokenizer} 输入 LLM,非文本模态由 \emph{Multimodal Encoder/Projector} 对齐后与文本融合\cite{tool-aug-mllm}。
\picHere{./assets/images/from-papers/tool-aug-mllm01.jpg}{0.8\linewidth}
{The overall architecture of MLLMs \cite{tool-aug-mllm}.}
{fig:tool-aug-mllm01}

\phead{Survey cues.}
更广义的“Agent AI”综述从“下一步具身动作预测(next-embodied action prediction)”出发,讨论外部知识、人类反馈与多传感输入在\emph{grounded} 场景中的作用\cite{agent-ai};面向 GUI 自动化的综述系统梳理了以 LLM 为中枢的框架、\emph{LAM}(large action models)、基准与指标\cite{llm-brained-gui};另有针对 (M)LLM-based GUI agents 的综述从\emph{感知—探索/知识—规划—交互}四组成对齐术语与挑战\cite{mllm-gui};在 OS 视角,\emph{OS Agents} 综述提出“环境/观测/动作—理解/规划/落地”的要素图谱\cite{os-agents};端侧推理综述为“低时延”语境提供工程参考\cite{ondevice-llm};“LLM×游戏智能体”的方法与基准的更广泛总览可参见近期 arXiv 工作\cite{game-agents-large-models}。

\phead{Industry/Community signals.}
作为动机与生态线索(非技术有效性论证),\emph{Neuro-sama} 的 \emph{AI streamer} 现象与相关开源生态(如 \emph{Open LLM VTuber}、\emph{Airi Project}、\emph{Kimjammer-Neuro})反映出该方向的关注与活跃\cite{neurosama_youtube,open_llm_vtuber,airi_project,kimjammer_neuro};平台侧量化数据亦显示较大受众体量:Neuro-sama 于 2025/01 创下 \textbf{Hype Train Level 111}(约 \textbf{85K} 付费订阅、约 \textbf{1.2M} bits),Twitch 2024 年总观看时长约 \textbf{18.5–20.8B} 小时、2024Q4 全行业约 \textbf{21B} 小时\cite{neurosama_hype_streamscharts_2025,pcgamesn_record_2025,vedal_twitchtracker,bloomberg_neurosama_2023,streamelements_state_2024,streamscharts_q4_2024_landscape}。

% ==========
% 1.3 (由原1.5前移)
% ==========
\subsection{Design Principles \& System Preview}
% > 文字级别的“系统流 + 原则”;详图与模块细节放 Methodology。此处仅做设计准则与一行流程。
\noindent\textbf{Design principles}:结构化输出(structured output)、可审计(auditability)、可复现(reproducibility)、低耦合(MCP-style 编排)。\;
\noindent\textbf{System preview}(一句话流程):\emph{screen/audio} $\rightarrow$ 轻量 \emph{VLM} $\rightarrow$ \emph{LLM/agentic}(planning/memory/reflection)$\rightarrow$ \emph{MCP-style} 技能注册/路由 $\rightarrow$ \emph{GUI 执行}(kb/mouse)$\rightarrow$ \emph{safety}(permissions, rollback, kill-switch)。\;
\noindent\textbf{默认工程姿态(不作实现承诺)}:single-flight(不并发大模型请求)、event-triggered(事件触发)、frame-window(3–5 帧堆叠)、text-first(可解析为结构化文本则优先文本路径),以降低延迟与方差。

% ==========
% 1.4 (位置后移)
% ==========
\subsection{Key Challenges}
% > 仅陈述挑战与已知现象;为 Evaluation Plan 与 Methodology 留空间。
(i)\textbf{长链路稳定性(long-horizon stability)}:在 GUI(GCC)通道下,错误累积与状态漂移更易放大;文献中的\emph{skills/反思/记忆}管线可缓解,但挑战仍存\cite{CRADLE};%
(ii)\textbf{视觉中心定位与记忆(vision-centric grounding \& memory)}:\emph{visual-only/continuous-space} 对定位、时机、视觉记忆与高层推理提出更高要求\cite{v-mage};%
(iii)\textbf{OOD 与协议一致(OOD \& protocol consistency)}:需要在\emph{过程生成}与\emph{变量可控}条件下比较架构/数据/后处理,减少不可比性\cite{Benchmarking-VLA-VLM};%
(iv)\textbf{提示方差与污染(prompt variance \& contamination)}:把“\emph{游戏→评测}”落到可复现协议需稳定交互回路并记录后处理\cite{lmgame-bench};%
(v)\textbf{无效动作与幻觉(invalid actions \& hallucination)}:\emph{结构化输出/约束解码}可降低 invalid action,但\emph{think–action mismatch} 等现象仍被报告\cite{ui-venus};%
(vi)\textbf{时延与交互体验(latency \& UX)}:实时伴随式场景强调\emph{voice RTT} 与帧到提示响应时间,需要与稳定性指标共同考量\cite{ORAK,lmgame-bench}。

% ==========
% 1.5 (原1.4)
% ==========
\subsection{Project Objectives \& Expected Deliverables}
% > Proposal写法:列“目标/交付物类别”,避免研究“贡献”口吻与过度承诺。
\noindent\textbf{Objectives.}
\begin{itemize}
  \item 实现一个\emph{以 GUI(GCC)为主}的\emph{伴随式}实时游戏助手原型,覆盖事件提示、策略建议与语音回路,面向低时延与可复现评测。
  \item 采用\emph{MCP-style} 的内部编排思想组织\emph{skills/macros、planning、memory、reflection},在不依赖应用专用 API 的前提下实现可插拔与可审计(auditability)。
  \item 明确一套小而可复现的评测要素(任务脚本与指标族),关注\emph{advice adoption}、\emph{voice RTT}、\emph{macro success} 等体验相关量。
\end{itemize}

\noindent\textbf{Expected Deliverables.}
\begin{itemize}
  \item \textbf{系统原型}:屏幕采集与轻量感知、agentic 模块、MCP-style 技能总线、GUI 执行器、基础安全护栏(权限/回滚/急停)。
  \item \textbf{评测脚本与配置}:可复现实验的任务脚本、指标计算与日志审计工具(含模块开关用于对照)。
  \item \textbf{使用文档与演示}:安装/运行说明、配置模板与演示视频;\textit{范围外}:针对具体应用/游戏的专用 API 适配与大规模模型微调。
\end{itemize}

% ==========
% 1.6 (新增,开题口径声明)
% ==========
\subsection{Assumptions \& Out-of-Scope}
% > 目的:提前声明工程与范围假设,降低评审对“宽口径”的不确定;与 2.7 部署现实呼应。
\phead{Working assumptions.}
默认采用“\emph{single-flight + event-triggered + frame-window(3–5 帧)+ text-first}”的工程姿态以降低时延与方差;评测记录\emph{post-processing} 与交互版本以保持协议一致。

\phead{Out-of-scope.}
不开展每个游戏/应用的专用 API 适配;不在本报告中承诺大规模端到端训练与数据采集;不涉及平台级增强权限(如 A11y/私有 DOM 钩子)的依赖。


% TODO: “虽然现在有类似于MAA、BetterGI、三月七小助手”这样的软件,但是他们主要是基于传统图像识别的自动化脚本,虽然能提升玩家体验,但是难以提供情感支持。