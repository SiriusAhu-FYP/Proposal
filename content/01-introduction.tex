\section{Introduction}
% 讲故事主线:先说“为什么需要/现在缺什么”,再说“我们选择什么路径与口径”,给出“系统预览与设计原则”,点出“关键挑战”,最后明确“目标/交付物/边界”。引用以代表性论文为证,不堆栈。

% ==========
% 1.1 Problem Setting & Motivation
% ==========
\subsection{Problem Setting \& Motivation}
% 作用:把“伴随式实时助手”的使用图景与缺口(统一动作接口、可复现实验)讲清楚,并以代表性基准/综述作为证据引入。
面向玩家的\emph{伴随式(companion-style)}实时助手,目标是在\textbf{不打断游戏流程}的前提下,持续提供\emph{事件提示(event spotting)}、\emph{策略建议(tactical guidance)}与\emph{语音回路(voice loop)}的低时延体验。近年的\emph{LLM$\times$游戏}与\emph{GUI 智能体}研究快速推进,但落地层面仍存在两类缺口:一是\textbf{动作接口不统一},大量工程依赖场景定制;二是\textbf{评测协议不一致},导致复现实验与横向对比困难。针对上述缺口,代表性工作正从\emph{统一协议/消融}与\emph{脚手架/污染控制}两侧收敛可复现方法学,为产品化路径提供了可操作依据\cite{ORAK,lmgame-bench,gp-agents}。

% ==========
% 1.2 Scope & Working Definitions
% ==========
\subsection{Scope \& Working Definitions}
% 作用:用“工作定义”定下本文的口径:执行通道(GCC)、编排方式(MCP-style)、输出形态(结构化输出/约束解码)、评测抓手(统一协议与事件型指标)。
本项目选择\textbf{以 GUI(GCC)为统一动作接口}(\emph{screen-in, keyboard/mouse-out} 的人类同态通道),并以\textbf{结构化输出(structured output)+ 约束解码(constrained decoding)}作为默认的动作生成与落地路径。代表性证据表明:在\emph{不依赖应用专用 API} 的前提下,通过“规划—技能整理(macro/skill)—自反思—记忆”的管线可跑通长链路任务;在真实平台上,\emph{截图} $\rightarrow$ \emph{结构化动作} 的端到端导航可复现,且\emph{合法动作映射(legal move constraint)}显著降低无效动作(invalid actions)\cite{CRADLE,ui-venus,Benchmarking-VLA-VLM}。内部组织上,本文采用\textbf{MCP-style 编排}(模块/技能/工具的注册与路由),用于支撑消融与替换;评测上参考\textbf{统一协议+脚手架} 的做法,并引入\emph{机会导向}的 OAS/RT/APO 三项工作定义以刻画伴随式体验\cite{ORAK,lmgame-bench}。

% ==========
% 1.3 Design Principles & System Preview
% ==========
\subsection{Design Principles \& System Preview}
% 作用:一句话系统流+设计原则;把“为什么这样设计”与前文证据钩住;引入端侧/工具增强作为工程抓手。

\noindent\textbf{Design principles.}
本文遵循四项原则:\textbf{结构化输出}(降低无效/便于审计)、\textbf{协议一致}(可复现/可消融)、\textbf{低耦合编排}(MCP-style,便于插拔 skills/tools)、\textbf{低时延}(面向 voice loop 与事件提示)。  

\noindent\textbf{System preview.}
系统流为:\emph{screen/audio} $\rightarrow$ 轻量\emph{VLM} 感知 $\rightarrow$ \emph{agentic}(planning/memory/reflection)$\rightarrow$ \emph{MCP-style} 技能/工具路由(含 OCR/检索/计算等 \emph{tool use})$\rightarrow$ \emph{GUI 执行}(kb/mouse)$\rightarrow$ \emph{safety}(确认/回滚/急停)。为降低端到端时延,部署层面将结合\emph{工具增强型 MLLM} 的分担思路与\emph{端侧推理}的量化/缓存策略作为工程抓手\cite{tool-aug-mllm,on-device-llm}。

% ==========
% 1.4 Key Challenges
% ==========
\subsection{Key Challenges}
% 作用:把“落到玩家侧真正难的事”说清楚,用通俗句式说明——为什么重要、怎么量、文献怎么佐证;为后文评测与设计留钩子(OAS/RT/APO、结构化输出、脚手架等)。

落到真实玩家场景,难点不是单一模型分数,而是\textbf{稳、准、快、可控}的整体体验。下面按\emph{可测}的挑战项列出。


\noindent\textbf{长链路稳定(long-horizon stability, GCC)。}
只走 GUI(GCC)时,误点与偏移会沿交互链放大,导致“走着走着就跑偏”。可用 \emph{pass@k}、回滚率(rollback)与 \textbf{APO}(attempts per opportunity)来量。\;实践显示,\emph{planning + skills(macro)+ reflection + memory} 的组合能缓解这类漂移,但并非万灵药\cite{CRADLE}。


\noindent\textbf{视觉定位与记忆(vision-centric grounding \& memory)。}
在仅视觉/连续空间设定下,\emph{定位/追踪/计数、时机控制、长期视觉记忆}是当前模型的短板,常见现象是“看见了但对不上位/时”。可按\textbf{OAS}(opportunity-normalized success)分机会类型统计,如“可拾取物/时间点/路径节点”等\cite{v-mage}。


\noindent\textbf{无效动作与幻觉(invalid actions \& think–action mismatch)。}
自由文本到动作容易“想得对、点错位”。\textbf{结构化输出 + 合法动作映射(legal move constraint)}能大幅压低 \emph{Invalid\%},并用 \emph{Brier/MAE} 看校准;训练侧可用“格式/类型/坐标/内容”四粒度奖励对齐执行细节\cite{Benchmarking-VLA-VLM,ui-venus}。


\noindent\textbf{OOD 与协议一致(OOD \& protocol consistency)。}
环境一换、版本一更,成绩就难对比。需要\emph{过程生成}(procedural generation)做分布外(OOD)评测、固定 \emph{post-processing} 与提示脚手架(scaffold)来控变量,并在\emph{统一协议}与 \emph{leaderboard/battle arena} 下报告 \emph{macro/micro} 指标\cite{Benchmarking-VLA-VLM,lmgame-bench,ORAK}。


\noindent\textbf{时延与交互体验(latency \& UX)。}
伴随式助手要“当下就回应”。关键是\textbf{RT}(reaction time per opportunity)与 \emph{voice RTT}(语音往返)。工程上需靠\emph{量化/剪枝/KV 缓存/流式解码}与\emph{端侧/端云协同}压时延,搭配单航班(single-flight)与可打断(barge-in)策略\cite{on-device-llm}。


\noindent\textbf{安全与鲁棒(safety \& robustness)。}
高风险动作必须“可确认、可回退、可追溯”。做法是\emph{权限白名单}、\emph{双确认}、\emph{影子执行(shadow execution)}与\emph{回滚/急停},并保存\emph{日志/审计}以定位 \emph{think–action mismatch}\cite{llm-brained-gui,mllm-gui,os-agents}。

% ==========
% 1.5 Project Objectives & Expected Deliverables
% ==========
\subsection{Project Objectives \& Expected Deliverables}
% 作用:以“产品向”列目标与交付物,不写研究承诺;指标对齐前文口径。

\noindent\textbf{Objectives.}
(i) 实现一个\textbf{以 GCC 为主}的\emph{伴随式}实时助手原型,覆盖事件提示、策略建议与语音回路;(ii) 采用\textbf{MCP-style 编排}组织 \emph{skills/macros、planning、memory、reflection},在不依赖应用专用 API 的前提下实现可插拔与可审计;(iii) 明确一套\textbf{小而可复现}的评测要素(任务脚本与指标族),关注 \emph{pass@k/TTC/Invalid\%/macro–micro} 以及 \emph{OAS/RT/APO} 等体验相关量\cite{ORAK,lmgame-bench}.  

\noindent\textbf{Expected deliverables.}
(a) \textbf{系统原型}:屏幕采集与轻量感知、agentic 模块、MCP-style 技能总线、GUI 执行器与基础安全护栏;(b) \textbf{评测脚本与配置}:可复现实验的任务脚本、指标计算与日志审计工具(含模块开关用于对照);(c) \textbf{使用文档与演示}:安装/运行说明、配置模板与演示视频。

% ==========
% 1.6 Assumptions & Out-of-Scope
% ==========
\subsection{Assumptions \& Out-of-Scope}
% 作用:提前声明工程与范围假设,降低评审对“宽口径”的不确定;与部署/评测呼应。

\noindent\textbf{Working assumptions.}
默认采用“\emph{single-flight + event-triggered + frame-window(3–5 帧)+ text-first}”的工程姿态以降低时延与方差;评测记录\emph{post-processing} 与环境版本以保持协议一致。  

\noindent\textbf{Out-of-scope.}
不开展\textbf{逐个应用/游戏的专用 API 适配};不在本报告中承诺\textbf{大规模端到端训练与数据采集};不依赖平台级增强权限(如 A11y/私有 DOM 钩子);\textbf{VLA 直出动作}仅作为评测对照而非默认路径\cite{Benchmarking-VLA-VLM}.

% TODO: “虽然现在有类似于MAA、BetterGI、三月七小助手”这样的软件,但是他们主要是基于传统图像识别的自动化脚本,虽然能提升玩家体验,但是难以提供情感支持。

% TODO: 模块化的skills/macro,方便社区开发适用于不同游戏的插件?

% TODO: 传给大模型的是处理后的图片?比如说用YOLO打标和位置?