\section{Introduction}
% > 说明:交代问题空间(companion-style game agent)、范围(scope)、研究缺口(gap)、本文立场与贡献(positioning & contributions)。涉及了1.1-1.5。

\subsection{Problem Setting \& Motivation}
% > 背景:实时游戏场景需要事件提示(event spotting)、策略建议(tactical guidance)、低延迟语音交互(voice loop)。
近年来,面向玩家的智能交互快速涌现:从\emph{AI游戏主播/虚拟角色}到\emph{LLM驱动的NPC/插件},社区与产业侧案例表明“\emph{大模型×游戏交互}”具备关注度与潜在影响(game changer potential)。然而,这些案例多为定制工程,缺乏统一接口与可复现实验协议。本文聚焦\emph{伴随式(companion-style)}实时助手,在\emph{统一动作接口}与\emph{低延迟体验}的约束下,探讨一条\emph{仅基于GUI(GCC)}的可复现实证路径。

\paragraph{Industry/Community Signals}
% > 产业/社区信号仅作动机,不作为学术证据。
除学术工作外,社区与产业侧的“AI$\times$游戏/直播”案例为本研究提供现实动机。例如 \emph{Neuro-sama} 及其开源复刻框架\cite{neurosama_youtube,open_llm_vtuber,kimjammer_neuro,airi_project},以及叙事解谜作品 \emph{AI2U: With You \textquotesingle Til The End}\cite{ai2u_game} 展示了“对话即操作”(dialogue-as-action)与高交互度(LLM-controlled NPCs)的设计可能性。这些案例\emph{不作为方法有效性的学术证据};本文的问题将落在\emph{统一动作接口(GUI/GCC)}、\emph{低延迟}与\emph{评测协议}上。

\subsection{Scope \& Working Definitions}
% > 定义术语:多模态(multimodal)、动作接口(GUI/GCC)、伴随式助手(companion-style)、短时托管(autopilot/macros)、评测术语(success rate, latency, advice adoption)。
我们采用\textbf{General Computer Control (GCC)} 的\emph{GUI}范式:\emph{screen-in, keyboard/mouse-out}的人类同态接口(human-homomorphic interface)。代表作 \emph{Cradle} 证明了在不依赖应用API的前提下完成长链路桌面/游戏任务的可行性\cite{tan2024cradle}。本文聚焦\emph{GUI(GCC)}通道;\emph{Model Context Protocol (MCP)} 被视为\emph{内部模块/技能编排}的思路(registration/orchestration),与输出通道无关。

% > 相关文献仅作背景介绍,不作采用/承诺;细节在第2章展开。
与本文话题相关的代表性工作包括:\emph{Orak}(统一评测/模块消融/对接协议)\cite{park2025orak};基于\emph{procedural generation} 研究OOD与变量可控性的评测方法\cite{guruprasad2025benchmarking};以及将真实游戏转化为可复现评测的\emph{lmgame-Bench}\cite{lmgame}。此外,\emph{UI-Venus} 强调\emph{纯截图(screenshot-only)}输入与结构化动作输出的端到端导航,不依赖planner或A11y树\cite{ui-venus};\emph{V\!-MAGE} 强调\emph{visual-only/continuous-space} 的视觉中心评测\cite{v-mage}。本文在相关工作部分对其进行系统梳理。

\paragraph{Screenshot-only navigation (GUI).}
% > 一句“纯背景”,避免承诺;数据仅作事实引用。
文献显示,在真实平台上,\emph{纯截图输入 + 结构化动作输出}也可实现端到端导航并取得有竞争力的结果(如AndroidWorld \emph{pass@1} 指标,以及ScreenSpot系列定位能力)\cite{ui-venus}。

\subsection{Key Challenges}
% > 长链路稳定性(long-horizon stability)、UI变化鲁棒(robustness)、延迟预算(latency budget)、权限安全(permissions/rollback)、跨游戏迁移(generality/portability)。
\todo{长链路稳定性、UI变化鲁棒、延迟预算、权限安全与回滚、跨游戏迁移;并注明仅依赖GUI控制带来的特定挑战(如确定性与重试策略)。}

\subsection{Our Positioning \& Contributions}
% > 只在此处陈述我们的立场与贡献;不在前文零散承诺。
\todo{(1) \textbf{GUI/GCC} 的伴随式助手设定;(2) \textbf{MCP} 作为内部“技能/工具总线”进行注册/编排(与输出通道无关);(3) 本文提出的评测协议与指标(advice adoption, voice RTT, macro success 等);(4) 论文结构与开源计划(如有)。}

\subsection{Design Principles \& System Preview}
% > 系统预告,细节放方法章节;避免在Intro承诺训练/微调/评测参数。
\todo{一句话系统流:\emph{screen/audio} → \emph{VLM} → \emph{LLM/agentic}(planning/memory/reflection)→ \emph{MCP-style}(技能注册/路由)→ \emph{GUI执行}(kb/mouse)→ \emph{safety}(permissions, rollback, kill-switch)。}
``
