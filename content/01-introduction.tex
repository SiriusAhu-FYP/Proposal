\section{Introduction}
% > 说明:交代问题空间(companion-style game agent)、范围(scope)、研究缺口(gap)、本文立场与贡献(positioning & contributions)。涉及了2.1-2.4和2.6-2.7。

\subsection{Problem Setting \& Motivation}
% > 背景:实时游戏场景需要事件提示(event spotting)、策略建议(tactical guidance)、低延迟语音交互(voice loop)。动机:VLM/VLA在真实桌面/游戏的落地与稳定性存在鸿沟。
近年来,面向玩家的智能交互快速涌现:从\emph{AI游戏主播/虚拟角色}到\emph{LLM驱动的NPC/插件},社区与产业侧案例表明“\emph{大模型×游戏交互}”具备显著关注度与潜在影响(game changer potential)。然而,这些案例多为定制工程,缺乏统一接口与可复现实验协议。本文聚焦\emph{伴随式(companion-style)}实时助手,在\emph{统一动作接口}与\emph{低延迟体验}的约束下,探索一条\emph{仅基于GUI(GCC)}即可落地、并具备可复现性的系统路径。

\paragraph{Industry/Community Signals}
% > 产业/社区信号仅作动机,不作为学术证据。
除学术工作外,社区与产业侧的“AI$\times$游戏/直播”案例为本研究提供现实动机。例如 \emph{Neuro-sama} 及其开源复刻框架\cite{neurosama_youtube,open_llm_vtuber,kimjammer_neuro,airi_project},以及叙事解谜作品 \emph{AI2U: With You \textquotesingle Til The End}\cite{ai2u_game} 展示了“对话即操作”(dialogue-as-action)与高交互度(LLM-controlled NPCs)的设计可能性。我们据此聚焦“\emph{游戏 + 大模型交互}”的可复现路径;这些案例\emph{不作为方法有效性的学术证据},研究问题将落在\emph{统一动作接口(GUI/GCC)}、\emph{低延迟}与\emph{评测协议}上。

\subsection{Scope \& Working Definitions}
% > 定义术语:多模态、动作接口(本文聚焦GUI/GCC)、伴随式助手、短时托管(autopilot/macros)、评测术语(success rate, latency, advice adoption)。
我们采用\textbf{General Computer Control (GCC)} 的\emph{GUI}范式:\emph{screen-in, keyboard/mouse-out}的人类同态接口(human-homomorphic interface)。代表作 \emph{Cradle} 证明了在不依赖应用API的前提下完成长链路桌面/游戏任务的可行性\cite{tan2024cradle}。在此基础上,\textbf{本文明确聚焦GUI},\emph{不进行应用API的单独适配}(尽管在可行时API可能进一步提升\emph{determinism \& efficiency},但其工程成本与维护负担超出本文范围)。相应地,我们\textbf{以 \emph{Model Context Protocol (MCP)} 作为内部“技能/工具总线”}:将\emph{GUI技能(skills/macros)}、\emph{规划/记忆/反思(planning/memory/reflection)}等模块以\emph{MCP风格}进行注册与编排,统一在GCC通道上执行。

% > 与评测设置相关的代表性工作引用与对齐(段末合并引用)。
作为与本文评测设置相关的代表性工作,\emph{Orak} 提供覆盖多类型真实电子游戏的统一基准,并以 \emph{MCP} 实现\emph{plug-and-play} 的代理—环境对接;其 \emph{Leaderboard/Battle Arena} 与\emph{agentic modules} 消融,为比较不同模块与输入模态提供了统一框架\cite{park2025orak}。我们参考其“统一评测—模块消融—可复现配置”的思路,但\emph{将输出接口限定为GUI(GCC)},并把 \emph{MCP} 用作\emph{内部技能编排协议}而非外部应用API适配层。

% > 程序生成/开放式评测参考,强调OOD与过程变量。
作为开放式环境下的系统评测参考,我们采用基于\emph{procedural generation}的统一框架来度量\emph{VLA/VLM}在多步轨迹与OOD设定中的表现,并将\emph{架构/数据/输出后处理}作为可控变量纳入对比\cite{guruprasad2025benchmarking}。

% Intro: Evaluation Preview / Scope 末尾(1–2句)
我们参考 \emph{lmgame-Bench} 的“游戏→可靠评测”思路:以统一 \emph{Gym-style API} 与轻量 \emph{perception/memory scaffolds} 控制\emph{提示方差(prompt variance)}与\emph{污染(contamination)},并度量跨游戏与多步任务的\emph{泛化(generalization)}表现;实现上我们仍\textbf{统一采用 GUI 执行},\emph{MCP-style} 仅用于内部\emph{技能编排(skill bus)}\cite{hu2505lmgame}.


\subsection{Key Challenges}
% > 长链路稳定性(long-horizon stability)、UI变化鲁棒(robustness)、延迟预算(latency budget)、权限安全(permissions/rollback)、跨游戏迁移(generality/portability)。
\todo{长链路稳定性、UI变化鲁棒、延迟预算、权限安全与回滚、跨游戏迁移;并注明我们仅依赖GUI控制带来的特定挑战(如确定性与重试策略)。}

\subsection{Our Positioning \& Contributions}
% > 立场:GUI/GCC;MCP用于内部技能/模块编排;不做API适配但讨论其潜在收益;提出伴随式评测协议。
\todo{(1) \textbf{GUI/GCC} 的伴随式助手原型;(2) 以 \textbf{MCP} 作为“技能/工具总线”统一规划/记忆/反思/技能组并落到GUI执行;(3) 提出\emph{advice adoption, voice RTT, macro success}等伴随式指标与统一评测设置;(4) 讨论API接入的潜在收益但不纳入本文范围。}

\subsection{Design Principles \& System Preview}
% > 系统预告:screen/audio → VLM → LLM/agentic modules → MCP-skill bus → GUI (kb/mouse) → safety guard。详图放方法章节。
\todo{一句话系统流:\emph{screen/audio} → \emph{VLM} → \emph{LLM/agentic}(planning/memory/reflection)→ \emph{MCP-skill bus}(技能注册/路由)→ \emph{GUI执行}(kb/mouse)→ \emph{safety}(permissions, rollback, kill-switch)。}

\subsection{Summary of Findings (Optional)}
% > 文献趋势与本文选择:文献存在向API/MCP确定性倾斜的趋势;本文基于成本/可移植性聚焦GUI,并以MCP做内部编排。
\todo{趋势:部分工作转向API/MCP以换取确定性与速度;选择:本文因工程成本与可迁移性\emph{聚焦GUI},并以\emph{MCP}完成内部技能/模块编排;未来可探索有限API接入的收益。}
