\section{Introduction}
% > 作用:用“产品向Proposal”的写法交代问题场景(companion-style agent)、范围(scope)、研究缺口(gap)、本文定位(positioning)。
% > 对标C.1模板:对应 1.1 Introduction/Background 与 1.2 Scope/Objectives 的“背景+范围”部分;不在此承诺实现/实验参数。

% ==========
% 1.1
% ==========
\subsection{Problem Setting \& Motivation}
% > 背景:实时游戏场景(real-time gaming)中的伴随式助手(companion-style)需求:事件提示(event spotting)、策略建议(tactical guidance)、低延迟语音交互(voice loop)。
% > Gap:目前多为定制工程,缺少统一接口与可复现实验协议;相关研究正探索统一评测/消融与污染控制等方向(纯回顾语气)。
近年来,面向玩家的智能交互快速涌现:从\emph{AI游戏主播/虚拟角色}到\emph{LLM驱动的NPC/插件},社区与产业案例显示“\emph{大模型×游戏交互}”具备关注度与潜在影响(game changer potential)。然而,这些实践多依赖场景定制,缺乏\emph{统一动作接口}与\emph{可复现实验协议};相应研究正在通过\emph{统一评测/模块消融}与\emph{污染控制/协议一致}等方法学加以弥合\cite{ORAK,lmgame}。本文讨论\emph{伴随式(companion-style)}实时助手,并将相关问题置于\emph{仅基于GUI(GCC)}与\emph{低延迟体验}的工作设定下加以界定。

\paragraph{Industry/Community Signals}
% > 仅作动机与生态线索(signals),不作为学术证据。
在社区层面,\emph{Neuro-sama} 的\emph{AI 游戏主播(AI streamer)}现象展示了大模型驱动的持续互动与情绪共鸣能力\cite{neurosama_youtube};同时,叙事解谜作品 \emph{AI2U: With You \textquotesingle Til The End} 体现了“\emph{对话即操作(dialogue-as-action)}”与高交互度(LLM-controlled NPCs)的设计潜力\cite{ai2u_game}。围绕该方向的开源复现与二次开发——如 \emph{Open LLM VTuber}、\emph{Airi Project}、\emph{Kimjammer-Neuro} 等——持续出现,反映出应用生态的活跃\cite{open_llm_vtuber,airi_project,kimjammer_neuro}。上述社区与产业案例\emph{仅作为动机与生态线索},并不作为本文方法有效性的学术证据。

作为市场信号,AI VTuber(以 \emph{Neuro-sama} 为代表)已在主流平台获得大规模关注:其在 2025 年 1 月创造 Twitch \emph{Hype Train} 世界纪录(Level 111,\,$\sim$85K 付费订阅、$\sim$1.2M bits),相关媒体与行业统计均有报道;其频道长期维持数十万粉丝规模与高并发活跃\cite{neurosama_hype_streamscharts_2025,pcgamesn_record_2025,vedal_twitchtracker,bloomberg_neurosama_2023}。同时,直播总体观看时长处于高位(如 Twitch 2024 年全年约 $18.5$–$20.8$\,B 小时;2024\,Q4 全行业约 $21$\,B 小时),显示“AI×直播/游戏”具备现实受众与可观市场体量\cite{streamelements_state_2024,streamscharts_q4_2024_landscape}.


% ==========
% 1.2
% ==========
\subsection{Scope \& Working Definitions}
% > 目的:界定本文术语与讨论范围,保持“工作定义(working definitions)”语气;不等同于方法承诺。
% > 关键词括号:GUI(GCC), MCP-style orchestration, screenshot-only, structured output, evaluation protocol。
本文将\textbf{动作接口}(action interface)的工作定义限定为 \textbf{General Computer Control (GCC)} 的\emph{GUI}范式:\emph{screen-in, keyboard/mouse-out} 的人类同态接口(human-homomorphic interface)。代表性工作 \emph{CRADLE} 报告了在不依赖应用 API 的前提下完成长链路桌面/游戏任务的可行性与系统结构(规划/技能整理/反思/记忆)\cite{CRADLE}。此外,\emph{Model Context Protocol (MCP)} 可作为\emph{内部模块/技能编排}(registration/orchestration)的通用思路,与具体的输出通道无直接绑定;“统一评测/消融”与\emph{plug-and-play} 思路可见 \emph{ORAK}\cite{ORAK};基于\emph{procedural generation} 的 OOD 方法学可见 \emph{Benchmarking-VLA-VLM}\cite{Benchmarking-VLA-VLM};将真实游戏“转化为可靠评测”的协议化实践可见 \emph{lmgame}\cite{lmgame}。在\emph{GUI} 场景中,\emph{UI-Venus} 强调\emph{纯截图(screenshot-only)}输入与\emph{结构化动作(structured output)}的端到端导航\cite{ui-venus};而 \emph{V\!-MAGE} 聚焦\emph{visual-only/continuous-space} 的视觉中心评测\cite{v-mage}。上述工作将在第二部分的相关研究中系统梳理。

\paragraph{Screenshot-only navigation (GUI).}
% > 事实性一句:仅指向“可行性”与“代表指标/任务”,避免扩展为实现承诺。
文献显示,在真实平台上,\emph{纯截图输入 + 结构化动作输出}亦可实现端到端导航并取得具有竞争力的结果(如 AndroidWorld 的 \emph{pass@1} 与 ScreenSpot 系列的屏幕定位任务)\cite{ui-venus}。

% > 通用MLLM结构示意:图随引用就近放置(你偏好“图在同一小节出现”)。
作为多模态大模型(MLLM)的通用结构示意,图~\ref{fig:tool-aug-mllm01} 展示了文本经 \emph{tokenizer} 输入 LLM、非文本模态经 \emph{Multimodal Encoder/Projector} 对齐后与文本融合的典型流程\cite{tool-aug-mllm}。

\picHere{./assets/images/from-papers/tool-aug-mllm01.jpg}{0.8\linewidth}
{The overall architecture of MLLMs \cite{tool-aug-mllm}.}
{fig:tool-aug-mllm01}

% > 将“综述线索”统一放到1.2末尾,保持1.1干净(不在动机里大段列文献)。
作为多模态交互智能体(Agent AI)的概念性综述,作者从“下一步具身动作预测(next-embodied action prediction)”出发,讨论外部知识、人类反馈与多传感输入在\emph{grounded} 场景下提升稳健性的作用\cite{agent-ai}。面向 GUI 自动化的综述则系统梳理了以 LLM 为“中枢”的 GUI 智能体在\emph{框架、训练数据/大动作模型(LAM)、评测基准与指标}方面的进展与挑战\cite{llm-brained-gui};另有面向 (M)LLM-based GUI agents 的综述从\emph{感知—探索/知识—规划—交互}四组成进行框架化对齐,并指出评测方法学与标准化的挑战\cite{mllm-gui};在更上层的 OS 范畴,\emph{OS Agents} 综述提出“环境/观测/动作—理解/规划/落地”的要素与能力图谱\cite{os-agents}。

% ==========
% 1.3
% ==========
\subsection{Key Challenges}
% > 仅列挑战与已知现象(reported issues),不提出解决方案;为Evaluation Plan与Methodology留出空间。
(i)\textbf{长链路稳定性(long-horizon stability)}:在\emph{GUI(GCC)}通道上,错误累积与状态漂移更易放大;文献通过\emph{skills/反思/记忆}等管线缓解,但挑战仍存\cite{CRADLE};%
(ii)\textbf{视觉中心定位与记忆(vision-centric grounding \& memory)}:\emph{visual-only/continuous-space} 设定对\emph{定位/时机/视觉记忆/高层推理}提出更高要求\cite{v-mage};%
(iii)\textbf{OOD与协议一致性(OOD \& protocol consistency)}:评测需在\emph{过程生成}与\emph{变量可控}的条件下比较架构/数据/后处理,减少不可比性\cite{Benchmarking-VLA-VLM};%
(iv)\textbf{提示方差与污染(prompt variance \& contamination)}:将“\emph{游戏→评测}”落到可复现协议需稳定交互回路并记录后处理\cite{lmgame};%
(v)\textbf{无效动作与幻觉(invalid actions \& hallucination)}:\emph{结构化输出/约束解码}可降低\emph{invalid action},但\emph{think–action mismatch} 等现象仍被报告\cite{ui-venus};%
(vi)\textbf{时延与交互体验(latency \& UX)}:实时伴随式场景强调语音往返(voice RTT)与帧到提示的响应时间,需与稳定性指标共同考量\cite{ORAK,lmgame}。

% ==========
% 1.4
% ==========
\subsection{Our Positioning \& Contributions}
% > Proposal写法:此处仅列“研究定位/预期交付物类别”的占位,不展开实现细节;与C.1模板的“Objectives/Expected Results”呼应。
\todo{(1) \textbf{研究定位}:GUI(GCC)下的伴随式助手设定;(2) \textbf{概念性模块化}:以 MCP-style 作为内部“技能/工具总线”进行注册/编排(与输出通道无关);(3) \textbf{评测要素}:统一的任务脚本与指标(advice adoption, voice RTT, macro success 等);(4) \textbf{预期交付物}:原型/评测脚本/文档(按C.1中的Expected Results表述)。}

% ==========
% 1.5
% ==========
\subsection{Design Principles \& System Preview}
% > 仅给“系统流”与设计原则的文字预告,详图与模块细节在Methodology中呈现;避免承诺训练/参数。
\todo{系统流一句话:\emph{screen/audio} → \emph{VLM} → \emph{LLM/agentic}(planning/memory/reflection)→ \emph{MCP-style}(技能注册/路由)→ \emph{GUI执行}(kb/mouse)→ \emph{safety}(permissions, rollback, kill-switch)。设计原则:结构化输出(structured output)、可审计(auditability)、可复现(reproducibility)。}
